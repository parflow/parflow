
%=============================================================================
%=============================================================================

\section{Installing ParFlow}
\label{Installing ParFlow}

\parflow{} is distributed as source code only and must be configured
and built (compiled) on each machine on which you would like to run
simulations.  \parflow{} uses the CMake for configuration.  Though we
will cover some basic guidelines for the installation of \parflow{}
here, many tips and tricks for building \parflow{} on a range of
systems may be found at the \parflow{} blog:
\file{http://parflow.blogspot.com}

For greater portability the \parflow{} build process allows separate
configuration and compilation of the simulator and support tools.
This separation allows easier porting to platforms where the
architecture is different on the nodes and the front-end.

These instructions are for building \parflow{} on a range of serial
and parallel Linux, Unix and OSX machines, including stand-alone
single and multi-core to large parallel clusters.  These instructions
do \emph{NOT} include compilation on Windows machines.

\parflow{} requires a Standard \emph{ANSI C} and \emph{FORTRAN 90/95}
compiler to build code. Many versions of \code{C} and \code{Fortran} are
compatible with \parflow{} (e.g. Intel or IBM).  However, \code{GCC} and \code{gFortran}, available for
free on almost every platform, are good options. They may be found at:
\begin{center}
\code{http://gcc.gnu.org/}  
\end{center}
and
\begin{center}
\code{http://gcc.gnu.org/wiki/GFortran}
\end{center}
\parflow{} also requires \code{TCL/TK} version 8.0 (or higher).  \code{TCL/TK} can be obtained from:
\begin{center}
\code{http://www.tcl.tk/}
\end{center}
These three packages are often pre-installed on most computers and
generally do not need to be installed by the user.  The following
steps are designed to take you through the process of installing
\parflow{} from a source distribution.  \parflow{} uses the \code{gnu}
package \code{autoconf} to create a configuration file for building
and installing the \parflow{} program.

\begin{enumerate}
\item {\bf Setup} \\

We will use the \file{~/pfdir} directory as the root directory for the
source, build and install directory in this user manual; you can use a
different directory if you wish.

Decide where you wish to install Parflow and
associated libraries. The following environment variable should be
set up in your \file{.profile}, \file{.cshrc}, or other file. Set
the environment variable \code{PARFLOW_DIR} to your chosen location
(if you are using \code{bash} or a bourne syntax shell):
\begin{display}\begin{verbatim}
export PARFLOW_DIR=~/pfdir/install
\end{verbatim}\end{display}
If you are using a \code{csh} like shell you will need the following in your
\file{.cshrc} file:
\begin{display}\begin{verbatim}
setenv PARFLOW_DIR ~/pfdir/install
\end{verbatim}\end{display}

%------------------------------

\item {\bf Extract the source}\\ Extract the source files from the
distribution compressed tar file or by cloning the repository from
the \parflow{} github repository.  This example assumes the
\file{parflow.tar.Z} file is in your home directory and you are
building it in a directory \file{~/parflow}.

\begin{display}\begin{verbatim}
mkdir ~/pfdir
cd ~/pfdir
tar -xvf ../parflow.tar.Z
\end{verbatim}\end{display}

%------------------------------

\item {\bf Build and install \parflow{}} with \cmake{}\\ This step builds
the \parflow{} and associated \pftools{} executables.  A \parflow{}
library is also built which can be used when \parflow{} is used as a
component of another simulation (\emph{e.g.}  WRF).

\cmake{} can be invoked using a command line only version \code{cmake}
or terminal based GUI \code{ccmake}.  Most of the example below will
use the command line version since it is easier to directly show the
command.  The \code{ccmake} version will show all the variables you
may set in an interactive GUI.

The following commands configure and build a sequenital version of
\parflow{}.  You can control build options for \parflow{} in the
\cmake{} configure step by adding other options to that command-line.
Note we build in a separate directory from the source.  This keeps the
source directory free of files created during the build.  Going back
to a clean state can be done by removing the build directory and
starting over.  This is fairly common \cmake{} practice.

Here we only set the directory to install in the \cmake{} command.

\begin{display}\begin{verbatim}
cd ~/pfdir
mkdir build
cd build
cmake ../parflow -DCMAKE_INSTALL_PREFIX=$(PARFLOW_DIR) 
make 
make install
\end{verbatim}\end{display}

For a list of \emph{all} the \cmake{} configure options, the easiest
method is to use the \code{ccmake} command:

\begin{display}\begin{verbatim}
ccmake ../parflow  -DCMAKE_INSTALL_PREFIX=$(PARFLOW_DIR)
\end{verbatim}\end{display}

\parflow{} defaults to building a sequential version; setting the
\code{PARFLOW_AMPS_LAYER} variable is required to build a parallel
version.  The easiest way to use the MPI version to use is to ensure
that the \code{mpicc} command is in your path and set the
\code{PARFLOW_AMPS_LAYER} to ``mpi1''.  The \code{PARFLOW_AMPS_LAYER}
variable controls \emph{AMPS} which stands for \emph{A}nother
\emph{M}essage \emph{P}assing \emph{S}ytem.  \emph{AMPS} is a flexible
message-passing adapter layer within \parflow{} that allows a common
code core to be quickly and easily adapted to different parallel
environments.

Here is an example of the simplest MPI configuration:

\begin{display}\begin{verbatim}
cmake ../parflow -DCMAKE_INSTALL_PREFIX=$(PARFLOW_DIR) -DPARFLOW_AMPS_LAYER=mpi1
\end{verbatim}\end{display}

\code{TCL} is required for building \parflow{}.  If \code{TCL} is not
installed in the system locations (\code{/usr} or \code{/usr/local})
you need to specify the path with the
\code{-DTCL_TCLSH=${PARFLOW_TCL_DIR}/bin/tclsh8.6} \code{cmake}
option.

\item {\bf Running a sample problem}\\ There is a test directory that
contains not only example scripts of \parflow{} problems but the
correct output for these scripts as well.  This may be used to test
the compilation process and verify that \parflow{} is installed
correctly.  If all went well a sample \parflow{} problem can be run
using:

\begin{display}\begin{verbatim}
cd $PARFLOW_DIR
cd test
tclsh default_single.tcl 1 1 1
\end{verbatim}\end{display}

Note that \code{PAFLOW_DIR} must be set for this to work and it
assumes tclsh is in your path.  Make sure to use the same \code{TCL}
as was used in the \cmake{} configure. The entire suite of test
cases may be run using \ctest{} to test a range of functionality in
\parflow{}.  This may be done by:
\begin{display}\begin{verbatim}
cd build
make test
\end{verbatim}\end{display}

\item {\bf Notes and other options:}\\ \parflow{} may be compiled with
a number of options using the configure script.  Some common options
are compiling \code{CLM} as in \cite{MM05,KM08a} to compile with
timing and optimization or to use a compiler other than \code{gcc}.
To compile with \code{CLM} add \code{-DPARFLOW_HAVE_CLM=ON} to the configure
line such as:

\begin{display}\begin{verbatim}
cmake ../parflow -DPARFLOW_AMPS_LAYER=mpi1 -DPARFLOW_HAVE_CLM=ON -DCMAKE_INSTALL_PREFIX=$(PARFLOW_DIR)) 
\end{verbatim}\end{display}

Other common options are:

\begin{itemize}
\item to include the \code{CLM} module:  \code{-DPARFLOW_HAVE_CLM=ON}
\item to include the \emph{SILO} 
library which provides greater file output formats that are compatible
with the \emph{VisIt} rendering package and must first be compiled 
separately (see below): \code{-DSILO_ROOT=$(PARFLOW_SILO_DIR)}
\item to include the \emph{HDF5} 
library which provides greater file output formats'' \code{-DHDF5_ROOT=$(PARFLOW_HDF5_DIR)}
\item to include the \emph{Hypre} library which provides greater solver
flexibility and options and also needs to be downloaded and built separately:
\code{-DHYPRE_ROOT=$(PARFLOW_HYPRE_DIR)}
\item to write a single, undistributed \parflow{} binary file:
\code{-DPARFLOW_AMPS_SEQUENTIAL_IO=true}
\item to write timing information in the log file: \code{-DPARFLOW_ENABLE_TIMING=true }
\end{itemize}

All these options combined in the configure line would look like:

\begin{display}\begin{verbatim}
cd build
cmake ../parflow \
  -DPARFLOW_AMPS_LAYER=mpi1 \
  -DPARFLOW_HAVE_CLM=ON \
  -DSILO_ROOT=$(PARFLOW_SILO_DIR) \
  -DHDF5_ROOT=$(PARFLOW_HDF5_DIR) \
  -DHYPRE_ROOT=$(PARFLOW_HYPRE_DIR) \
  -DCMAKE_INSTALL_PREFIX=$(PARFLOW_DIR)) 
make 
make install
\end{verbatim}\end{display}

To enable detailed timing of the performance of several different
components within \parflow{} use the \code{--enable-timing} option.
The standard \code{CMAKE_BUILD_TYPE} variable controls if a debug or
release (with optimization) is built.
\code{-DCMAKE_BUILD_TYPE=RELEASE} will build a release version.

It is often desirable to use different C and F90/95 compilers (such as
\emph{Intel} or \emph{Portland Group}) to generate optimized code for
specific architectures or simply personal preference.  To change
compilers, set the \code{CC}, \code{FC} and \code{F77} variables
(these may include a path too).  For example, to change to the
\emph{Intel} compilers in the \code{bash} shell:
\begin{display}\begin{verbatim}
export CC=icc
export FC=ifort
export F77=ifort
\end{verbatim}\end{display}

\item {\bf Build and install \parflow{}} with \code{GNU autoconf}\\

In addition configuration with \code{cmake}, \parflow{} has deprecated
support for \code{GNU autoconf}. \code{GNU autoconf} support will be
maintained while the \code{cmake} support is fully tested.  Support
will be removed in a future release.  Please use \code{cmake} and
report bugs.
  
This step builds the \parflow{} library and executable that runs on
a serial or parallel machine.  The library is used when
 \parflow{} is used as a component of another simulation (\emph{e.g.}
 WRF).  

\begin{display}\begin{verbatim}
cd $PARFLOW_DIR
cd pfsimulator
./configure --prefix=$PARFLOW_DIR --with-amps=mpi1
make 
make install
\end{verbatim}\end{display}

This will build a parallel version of /parflow{} using the MPI1 
libraries but no other options (a very basic installation with few features commonly used).
You can control build options for /parflow{} in the configure step by adding other 
options to that command-line. For a list of \emph{all} the configure options, use
\begin{display}\begin{verbatim}
./configure --help 
\end{verbatim}\end{display} to list them.  Note that \parflow{} defaults to building a sequential version so
\code{--with-amps} is needed when building for a parallel computer.
  You can explicitly specify the path to the MPI to use with the
 \code{--with-mpi} option to configure.  This controls \emph{AMPS}
 which stands for \emph{A}nother \emph{M}essage \emph{P}assing
 \emph{S}ytem.  \emph{AMPS} is a flexible message-passing layer within
 \parflow{} that allows a common code core to be quickly and easily
 adapted to different parallel environments. 

\item {\bf Build and install pftools}\\ 
\code{pftools} is a package of utilities and a \code{TCL} library that
 is used to setup and postprocess Parflow files.  The input files to
 \parflow{} are \code{TCL} scripts so \code{TCL} must be installed on
 the system.

The first command will build \parflow{} and the bundled tools and
 install them in the \code{\$PARFLOW_DIR} directory.  The second
 command will build and install the documentation.  A bare-bones
 configure and build looks like:

\begin{display}\begin{verbatim}
cd $PARFLOW_DIR
cd pftools
./configure --prefix=$PARFLOW_DIR --with-amps=mpi1
make 
make install
make doc_install
\end{verbatim}\end{display}

Note that \code{pftools} is \emph{NOT} parallel but some options for
how files are written are based on the communication layer so pftools
needs to have the same options that were used to build the \parflow{}
library.

If \code{TCL} is not installed in the system locations (\code{/usr} or
\code{/usr/local}) you need to specify the path with the
\code{--with-tcl=<PATH>} configure option.

See \file{./configure --help} for additional configure options for \code{pftools}.

\item {\bf Running a sample problem}\\ There is a test directory that
  contains not only example scripts of \parflow{} problems but the
  correct output for these scripts as well.  This may be used to test
  the compilation process and verify that \parflow{} is installed
  correctly.  If all went well a sample \parflow{} problem can be run
  using:

\begin{display}\begin{verbatim}
cd $PARFLOW_DIR
cd test
tclsh default_single.tcl 1 1 1
\end{verbatim}\end{display}

Note that \code{PAFLOW_DIR} must be set for this to work and it assumes tclsh
is in your path.  Make sure to use the same \code{TCL} as was used in the
\code{pftools} configure. The entire suite of test cases may be run at once to test a range of functionality in \parflow{}.  This may be done by:
\begin{display}\begin{verbatim}
cd $PARFLOW_DIR
cd test
make check
\end{verbatim}\end{display}

\item {\bf Notes and other options:}\\ \parflow{} may be compiled with
 a number of options using the configure script.  Some common options
 are compiling \code{CLM} as in \cite{MM05,KM08a} to compile with
 timing and optimization or to use a compiler other than \code{gcc}.
  To compile with \code{CLM} add \code{--with-clm} to the configure
 line such as:
\begin{display}\begin{verbatim}
./configure --prefix=$PARFLOW_DIR --with-amps=mpi1 --with-clm
\end{verbatim}\end{display}

Common options are:

\begin{itemize}
\item to include the \code{CLM} module:  \code{--with-clm}
\item to include the \emph{SILO} 
library which provides greater file output formats that are compatible
with the \emph{VisIt} rendering package and must first be compiled 
separately (see below): \code{--with-silo=$SILO_DIR}
\item to include the \emph{Hypre} library which provides greater solver
flexibility and options and also needs to be downloaded and built separately:
\code{--with-hypre=$HYPRE_DIR}
\item to write a single, undistributed \parflow{} binary file:
\code{--with-amps-sequential-io}
\item to write timing information in the log file: \code{--enable-timing }
\end{itemize}

All these options combined in the configure line would look like:

\begin{display}\begin{verbatim}
cd $PARFLOW_DIR
cd pfsimulator
./configure --prefix=$PARFLOW_DIR --with-amps=mpi1 --with-clm 
--enable-timing --with-silo=$SILO_DIR --with-hypre=$HYPRE_DIR
--with-amps-sequential-io
make 
make install
\end{verbatim}\end{display}

\code{pftools} needs to be compiled and built with compatible options
that correspond to \parflow{}.  For the options above, \code{pftools} would
be built as follows:

\begin{display}\begin{verbatim}
cd $PARFLOW_DIR
cd pftools
./configure --prefix=$PARFLOW_DIR --with-amps=mpi1 --with-silo=$SILO_DIR
--with-amps-sequential-io

make 
make install
make doc_install
\end{verbatim}\end{display}

Note that \code{CLM} and \emph{Hypre} are not used by \code{pftools} and
those options do not need to be included, however the file formats (\emph{SILO} 
and single file \code{PFB}) are very important and need to match exactly what is
specified in \code{pfsimulator}.

To enable detailed timing of the performance of several different
components within \parflow{} use the \code{--enable-timing} option.
To use compiler optimizations use the \code{--enable-opt=STRING} where
the \code{=STRING} is an optional flag to specify the level and type
of optimization.

IMPORTANT NOTE: Optimization and debugging are controlled independent of one
another.  So to compile with optimization and no debugging you need to
specify both \code{--enable-opt=STRING} AND \code{--disable-debug}.

It is often desirable to use different C and F90/95 compilers 
(such as \emph{Intel} or \emph{Portland Group}) to match hardware specifics,
for performance reasons or simply personal preference.  To change compilers,
set the \code{CC}, \code{FC} and \code{F77} variables (these may include a path too). 
For example, to change to the \emph{Intel} compilers in c-shell:
\begin{display}\begin{verbatim}
setenv CC icc
setenv FC ifort
setenv F77 ifort
\end{verbatim}\end{display}
\end{enumerate}

\subsection{External Libraries}
\label{External Libraries}
Many of the features of \parflow{} use a file structure called Silo.  
Silo is a free, open-source, format detailed at:
\begin{center}
\code{https://wci.llnl.gov/codes/silo/}
\end{center}
Support for Silo is integrated into \parflow{} but the Silo libraries 
must be built separately and then linked into \parflow{} during the 
build and configure process.   This may be done using the \code{--with-silo=PATH} 
where the \code{PATH} is the location of the Silo libraries.

Some features of \parflow{} need to call the solver package \emph{Hypre} externally.  
These include the command options {\bf PFMG}, {\bf SMG} and  {\bf PFMGOctree}. \emph{Hypre} is a
free, open-source, library detailed at:
\begin{center}
\code{https://computation.llnl.gov/casc/hypre/software.html}
\end{center}
Support for \emph{Hypre} 2.4.0b or later is integrated into \parflow{} but the libraries
must be built separately and then linked into \parflow{} during the build and configure process. 
This may be done using the \code{--with-hypre=PATH} where the \code{PATH} is the location of the
\emph{Hypre} libraries.
