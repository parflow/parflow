\begin{thebibliography}{999}

\bibitem{Ajami14} 
Ajami, H., McCabe, M.F., Evans, J.P. and Stisen, S. (2014). Assessing the impact of model spin-up on surface water-groundwater interactions using an integrated hydrologic model. {\em Water Resources Research} {\bf 50} 2636¿2656, doi:10.1002/2013WR014258.


\bibitem{Ajami2015}
Ajami,H., M.F. McCabe, and J.P. Evans (2015).
\newblock Impacts of model initialization on an integrated surface water--groundwater model.
\newblock {\em Hydrological Processes}, 29(17):3790--3801.

\bibitem{Ajami2018}
Ajami, H.; Sharma, A. (2018). Disaggregating Soil Moisture to Finer Spatial Resolutions: A Comparison of Alternatives. {\em Water Resour. Res.} , {\bf 54 (11)}, 9456–9483. https://doi.org/10.1029/2018WR022575.

\bibitem{Ashby-Falgout90}
Ashby, S.F. and Falgout,R.~D. (1996).
\newblock A parallel multigrid preconditioned conjugate gradient algorithm for
  groundwater flow simulations.
\newblock {\em Nuclear Science and Engineering}, {\bf 124}:145--159.

\bibitem{AM10}
Atchley, A. and Maxwell, R.M. (2011). Influences of subsurface heterogeneity and vegetation cover on soil moisture, surface temperature, and evapotranspiration at hillslope scales. {\em Hydrogeology Journal} doi:10.1007/s10040-010-0690-1.

\bibitem{Atchley13a} 
Atchley, A.L., Maxwell, R.M. and Navarre-Sitchler, A.K. (2013). Human Health Risk Assessment of CO 2 Leakage into Overlying Aquifers Using a Stochastic, Geochemical Reactive Transport Approach. {\em Environmental Science and Technology} {\bf 47} 5954--5962, doi:10.1021/es400316c.

\bibitem{Atchley13b} 
Atchley, A.L., Maxwell, R.M. and Navarre-Sitchler, A.K. (2013). Using streamlines to simulate stochastic reactive transport in heterogeneous aquifers: Kinetic metal release and transport in CO2 impacted drinking water aquifers. {\em Advances in Water Resources} {\bf 52} 93--106, doi:10.1016/j.advwatres.2012.09.005.

\bibitem{Atchley2017}
Atchley, A. L.; Kinoshita, A. M.; Lopez, S. R.; Trader, L.; Middleton, R. (2018). Simulating Surface and Subsurface Water Balance Changes Due to Burn Severity. {\em Vadose Zo. J.} {\bf 17 (1)}. https://doi.org/10.2136/vzj2018.05.0099.

\bibitem{Bhaskar2015}
Bhaskar,A.S., C. Welty, R.M. Maxwell, and A.J. Miller (2015).
\newblock Untangling the effects of urban development on subsurface storage in
  baltimore.
\newblock {\em Water Resources Research}, 51(2):1158--1181.

\bibitem{Baatz2017}
Baatz, D.; Kurtz, W.; Hendricks Franssen, H. J.; Vereecken, H.; Kollet, S. J. (2017). Catchment Tomography - An Approach for Spatial Parameter Estimation. {\em Adv. Water Resour.} {\bf 107}, 147–159. https://doi.org/10.1016/j.advwatres.2017.06.006.

\bibitem{Barnes2015}
Barnes,M. C.~Welty, and A.~Miller (2015).
\newblock Global topographic slope enforcement to ensure connectivity and
  drainage in an urban terrain.
\newblock {\em Journal of Hydrologic Engineering}, 0(0):06015017.

\bibitem{Barnes2018}
Barnes, M. L.; Welty, C.; Miller, A. J. (2018). Impacts of Development Pattern on Urban Groundwater Flow Regime. {\em Water Resour. Res.} {\bf 54 (8)}, 5198–5212. https://doi.org/10.1029/2017WR022146.

\bibitem{Baroni2019}
Baroni, G.; Schalge, B.; Rakovec, O.; Kumar, R.; Schüler, L.; Samaniego, L.; Simmer, C.; Attinger, S. A (2019). Comprehensive Distributed Hydrological Modeling Intercomparison to Support Process Representation and Data Collection Strategies. {\em Water Resour. Res.} {\bf 990–1010}. https://doi.org/10.1029/2018WR023941.

\bibitem{Bearup2016}
Bearup,L.A. R.M. Maxwell and J.E. McCray (2016).
\newblock Hillslope response to insect-induced land-cover change: an integrated
  model of end-member mixing.
\newblock {\em Ecohydrology},
\newblock ECO-15-0202.R1.

\bibitem{Beisman2015}
Beisman,J.J., R.~M. Maxwell, A.~K. Navarre-Sitchler, C.~I. Steefel, and
S.~Molins (2015).
\newblock Parcrunchflow: an efficient, parallel reactive transport simulation
  tool for physically and chemically heterogeneous saturated subsurface
  environments.
\newblock {\em Computational Geosciences}, 19(2):403--422.


\bibitem{Burger12} 
B{\"u}rger, C.M., Kollet, S., Schumacher, J. and Bösel, D. (2012). Introduction of a web service for cloud computing with the integrated hydrologic simulation platform ParFlow. {\em Computers and Geosciences} {\bf 48} 334--336, doi:10.1016/j.cageo.2012.01.007.

\bibitem{Condon13a} 
Condon, L.E. and Maxwell, R.M. (2013). Implementation of a linear optimization water allocation algorithm into a fully integrated physical hydrology model. {\em Advances in Water Resources} {\bf 60} 135--147, doi:10.1016/j.advwatres.2013.07.012.

\bibitem{Condon13b} 
Condon, L.E., Maxwell, R.M. and Gangopadhyay, S. (2013). The impact of subsurface conceptualization on land energy fluxes. {\em Advances in Water Resources} {\bf 60} 188--203, doi:10.1016/j.advwatres.2013.08.001.

\bibitem{Condon14a} 
Condon, L.E. and Maxwell, R.M. (2014). Feedbacks between managed irrigation and water availability: Diagnosing temporal and spatial patterns using an integrated hydrologic model. {\em Water Resources Research} {\bf 50} 2600--2616, doi:10.1002/2013WR014868.

\bibitem{Condon14b} 
Condon, L.E. and Maxwell, R.M. (2014). Groundwater-fed irrigation impacts spatially distributed temporal scaling behavior of the natural system: a spatio-temporal framework for understanding water management impacts. {\em Environmental Research Letters} {\bf 9} 1--9, doi:10.1088/1748-9326/9/3/034009.

\bibitem{Condon2015}
 Condon, L.E., A.~S. Hering, and R.~M. Maxwell (2015).
\newblock Quantitative assessment of groundwater controls across major \{US\}
  river basins using a multi-model regression algorithm.
\newblock {\em Advances in Water Resources}, 82:106 -- 123.

\bibitem{Condon2015a}
Condon, L.E., and R.~M. Maxwell (2015).
\newblock Evaluating the relationship between topography and groundwater using
  outputs from a continental-scale integrated hydrology model.
\newblock {\em Water Resources Research}, 51(8):6602--6621.


\bibitem{Condon2017}
Condon, L. E.; Maxwell, R. M. (2017). Systematic Shifts in Budyko Relationships Caused by Groundwater Storage Changes. {\em Hydrol. Earth Syst. Sci.} {\bf 21 (2)}, 1117–1135. https://doi.org/10.5194/hess-21-1117-2017.


\bibitem{Condon2019a}
Condon, L. E.; Maxwell, R. M. (2019). Simulating the Sensitivity of Evapotranspiration and Streamflow to Large-Scale Groundwater Depletion. {\em Sci. Adv.}. {\bf 5} (6). https://doi.org/10.1126/sciadv.aav4574.

\bibitem{Condon2019b}
Condon, L. E.; Maxwell, R. M. (2019). Modified priority flood and global slope enforcement algorithm for topographic processing in physically based hydrologic modeling applications. {\em Computers and Geosciences}, {\bf 126}(February), 73–83. https://doi.org/10.1016/j.cageo.2019.01.020.

%\bibitem{Cui2014}
%Cui,Z. Claire Welty, and Reed~M. Maxwell.
%\newblock Modeling nitrogen transport and transformation in aquifers using a
%  particle-tracking approach.
%\newblock {\em Computers \& Geosciences}, 70:1 -- 14, 2014.


\bibitem{Cui14}
Cui, Z., Welty, C. and Maxwell, R.M. (2014). 
\newblock Modeling nitrogen transport and transformation in aquifers using a particle-tracking approach. {\em Computers and Geosciences} {\bf 70} 1--14, doi:10.1016/j.cageo.2014.05.005.

\bibitem{Dai03}
Dai, Y., X. Zeng, R.E. Dickinson, I. Baker, G.B. Bonan, M.G. Bosilovich, A.S. Denning, P.A. Dirmeyer, P.R., G. Niu, K.W. Oleson, C.A. Schlosser and Z.L. Yang (2003). The common land model. {\em The Bulletin of the American Meteorological Society} {\bf 84}(8):1013--1023.

\bibitem{Danesh-Yazdi2018}
Danesh-Yazdi, M.; Klaus, J.; Condon, L. E.; Maxwell, R. M. (2018). Bridging the Gap between Numerical Solutions of Travel Time Distributions and Analytical Storage Selection Functions. {\em Hydrol. Process.}.{\bf 32} (8), 1063–1076. https://doi.org/10.1002/hyp.11481.

\bibitem{DMC10}
Daniels, M.H., Maxwell, R.M., Chow, F.K. (2010). An algorithm for flow direction enforcement using subgrid-scale stream location data, {\em Journal of Hydrologic Engineering} {\bf 16} 677--683, doi:10.1061/(ASCE)HE.1943-5584.0000340.

\bibitem{dBRM08}
de Barros, F.P.J., Rubin, Y. and Maxwell, R.M. (2009). The concept of comparative information yield curves and their application to risk-based site characterization. {\em Water Resources Research} 45, W06401, doi:10.1029/2008WR007324.

\bibitem{deRooij13} 
de Rooij, R., Graham, W. and Maxwell, R.M. (2013). A particle-tracking scheme for simulating pathlines in coupled surface-subsurface flows. {\em Advances in Water Resources} {\bf 52} 7--18, doi:10.1016/j.advwatres.2012.07.022.

\bibitem{EW96}
Eisenstat, S.C. and Walker, H.F. (1996).
\newblock Choosing the forcing terms in an inexact newton method.
\newblock {\em SIAM J. Sci. Comput.}, {\bf 17}(1):16--32.

\bibitem{Engdahl2015}
	Nicholas~B. Engdahl and Reed~M. Maxwell (2015).
\newblock Quantifying changes in age distributions and the hydrologic balance
  of a high-mountain watershed from climate induced variations in recharge.
\newblock {\em Journal of Hydrology}, 522:152 -- 162.

\bibitem{Fang2015}
	Zhufeng Fang, Heye Bogena, Stefan Kollet, Julian Koch, and Harry Vereecken (2015).
\newblock Spatio-temporal validation of long-term 3d hydrological simulations
  of a forested catchment using empirical orthogonal functions and wavelet
  coherence analysis.
\newblock {\em Journal of Hydrology}, 529, Part 3:1754 -- 1767.

\bibitem{Fang2016}
Fang, Y., L. R. Leung, Z. Duan, M. S. Wigmosta, R. M. Maxwell, J. Q. Chambers, and J. Tomasella (2016). Influence of landscape heterogeneity on water available to tropical forests in an Amazonian catchment and implications for modeling drought response, {\em J. Geophys. Res. Atmos.}, {\bf 122}, 8410–8426, doi:10.1002/2017JD027066.


\bibitem{Fang2017}
Fang, Yilin ; Leung, L. Ruby; Duan, Zhuoran; Wigmosta, Mark S.; Maxwell, Reed M.; Chambers, Jeffrey Q.; Tomasella, J. (2017). Journal of Geophysical Research : Atmospheres. {\em J. Geophys. Res. Atmos.} {/bf 122}, 3672–3685. https://doi.org/10.1002/2016JD025676.

\bibitem{FM10}
Ferguson, I.M. and Maxwell, R.M. (2010). Role of groundwater in watershed response and land surface feedbacks under climate change. {\em Water Resources Research} 46, W00F02, doi:10.1029/2009WR008616.

\bibitem{Ferg11} 
Ferguson, I.M. and Maxwell, R.M. (2011). Hydrologic and land¿energy feedbacks of agricultural water management practices. {\em Environmental Research Letters} {\bf 6} 1--7, doi:10.1088/1748-9326/6/1/014006.

\bibitem{Ferg12} 
Ferguson, I.M. and Maxwell, R.M. (2012). Human impacts on terrestrial hydrology: climate change versus pumping and irrigation. {\em Environmental Research Letters} {\bf 7} 1--8, doi:10.1088/1748-9326/7/4/044022.

\bibitem{Ferguson2016}
Ferguson, I. M.; Jefferson, J. L.; Maxwell, R. M.; Kollet, S. J. (2016). Effects of Root Water Uptake Formulation on Simulated Water and Energy Budgets at Local and Basin Scales. {\em Environ. Earth Sci.} {\bf 75 (4)}, 1–15. https://doi.org/10.1007/s12665-015-5041-z.

\bibitem{Forrester2018}
Forrester, M. M.; Maxwell, R. M.; Bearup, L. A.; Gochis, D. J. (2018). Forest Disturbance Feedbacks From Bedrock to Atmosphere Using Coupled Hydrometeorological Simulations Over the Rocky Mountain Headwaters.{\em J. Geophys. Res. Atmos.} {\bf 123 (17)}, 9026–9046. https://doi.org/10.1029/2018JD028380.

\bibitem{FWP95}
Forsyth, P.A., Wu, Y.S. and Pruess, K. (1995).
\newblock Robust Numerical Methods for Saturated-Unsaturated Flow with Dry Initial Conditions.
\newblock {\em Advances in Water Resources}, {\bf 17}:25--38.

\bibitem{Foster2016}
Foster, L. M.; Bearup, L. A.; Molotch, N. P.; Brooks, P. D.; Maxwell, R. M. (2016). Energy Budget Increases Reduce Mean Streamflow More than Snow-Rain Transitions: Using Integrated Modeling to Isolate Climate Change Impacts on Rocky Mountain Hydrology. {\em Environ. Res. Lett.} {\bf 11 (4)}. https://doi.org/10.1088/1748-9326/11/4/044015.


\bibitem{Foster2019}
Foster, L. M.; Maxwell, R. M. (2019). Sensitivity Analysis of Hydraulic Conductivity and Manning’s n Parameters Lead to New Method to Scale Effective Hydraulic Conductivity across Model Resolutions. {\em Hydrol. Process.} {\bf 33 (3)}, 332–349. https://doi.org/10.1002/hyp.13327.


\bibitem{FFKM09}
Frei, S., Fleckenstein, J.H., Kollet, S.J. and Maxwell, R.M. (2009). Patterns and dynamics of river-aquifer exchange with variably-saturated flow using a fully-coupled model. {\em Journal of Hydrology} 375(3-4), 383--393, doi:10.1016/j.jhydrol.2009.06.038.

\bibitem{Gebler2017}
Gebler, S.; Hendricks Franssen, H. J.; Kollet, S. J.; Qu, W.; Vereecken, H. (2017). High Resolution Modelling of Soil Moisture Patterns with TerrSysMP: A Comparison with Sensor Network Data. {\em J. Hydrol.} {\bf 547}, 309–331. https://doi.org/10.1016/j.jhydrol.2017.01.048.


\bibitem{Gilbert2016}
Gilbert, J. M.; Jefferson, J. L.; Constantine, P. G.; Maxwell, R. M. (2016). Global Spatial Sensitivity of Runoff to Subsurface Permeability Using the Active Subspace Method. {\em Adv. Water Resour.} {\bf 92}, 30–42. https://doi.org/10.1016/j.advwatres.2016.03.020.


\bibitem{Gilbert2017a}
Gilbert, J. M.; Maxwell, R. M. (2017). Examining Regional Groundwater-Surface Water Dynamics Using an Integrated Hydrologic Model of the San Joaquin River Basin. {\em Hydrol. Earth Syst. Sci.} {\bf 21 (2)}, 923–947. https://doi.org/10.5194/hess-21-923-2017.

\bibitem{Gilbert2017b}
Gilbert, J. M.; Maxwell, R. M.; Gochis, D. J. (2017). Effects of Water-Table Configuration on the Planetary Boundary Layer over the San Joaquin River Watershed, California. {\em J. Hydrometeorol.} {\bf 18 (5)}, 1471–1488. https://doi.org/10.1175/JHM-D-16-0134.1.


\bibitem{Gilbert2018}
Gilbert, J. M.; Maxwell, R. M. (2018). Contrasting Warming and Drought in Snowmelt-Dominated Agricultural Basins: Revealing the Role of Elevation Gradients in Regional Response to Temperature Change. {\em Environ. Res. Lett.} {\bf 13 (7)}. https://doi.org/10.1088/1748-9326/aacb38.


\bibitem{Gou2018}
Gou, S.; Miller, G. R.; Saville, C.; Maxwell, R. M.; Ferguson, I. M. (2018). Simulating Groundwater Uptake and Hydraulic Redistribution by Phreatophytes in a High-Resolution, Coupled Subsurface-Land Surface Model. {\em Adv. Water Resour.} {\bf 121} (August 2017), 245–262. https://doi.org/10.1016/j.advwatres.2018.08.008.

\bibitem{Haverkamp-Vauclin81}
Haverkamp, R. and Vauclin, M. (1981).
\newblock A comparative study of three forms of the {R}ichard equation used for
  predicting one-dimensional infiltration in unsaturated soil.
\newblock {\em Soil Sci. Soc. of Am. J.}, {\bf 45}:13--20.

\bibitem{Hein2019}
Hein, A.; Condon, L.; Maxwell, R. (2019). Unravelling the Impacts of Precipitation, Temperature and Land-Cover Change for Extreme Drought over the North American High Plains. {\em Hydrol. Earth Syst. Sci. Discuss.} {\bf 1–30}. https://doi.org/10.5194/hess-2018-485.

\bibitem{Jefferson2015a}
	Jennifer~L. Jefferson and Reed~M. Maxwell (2015).
\newblock Evaluation of simple to complex parameterizations of bare ground
  evaporation.
\newblock {\em Journal of Advances in Modeling Earth Systems}, 7(3):1075--1092.

\bibitem{Jefferson2015}
Jennifer~L. Jefferson, James~M. Gilbert, Paul~G. Constantine, and Reed~M.
Maxwell (2015).
\newblock Active subspaces for sensitivity analysis and dimension reduction of
  an integrated hydrologic model.
\newblock {\em Computers \& Geosciences}, 83:127 -- 138.

\bibitem{Jefferson2017}
Jefferson, J. L.; Maxwell, R. M.; Constantine, P. G. (2017). Exploring the Sensitivity of Photosynthesis and Stomatal Resistance Parameters in a Land Surface Model. {\em J. Hydrometeorol.} {\bf 18 (3)}, 897–915. https://doi.org/10.1175/jhm-d-16-0053.1.


\bibitem{Jones-Woodward01}
Jones, J.E. and Woodward, C.S. (2001).
\newblock Newton-krylov-multigrid solvers for large-scale, highly heterogeneous, variably saturated flow problems.
\newblock {\em Advances in Water Resources}, {\bf 24}:763--774.


\bibitem{Keune2016}
Keune, J.; Gasper, F.; Goergen, K.; Hense, A.; Shrestha, P.; Sulis, M.; Kollet, S. (2016). Studying the Influence of Groundwater Representations on Land Surface-Atmosphere Feedbacks during the European Heat Wave in 2003. {\em J. Geophys. Res.} {\bf 121 (22)}, 13,301-13,325. https://doi.org/10.1002/2016JD025426.


\bibitem{Keune2018}
Keune, J.; Sulis, M.; Kollet, S.; Siebert, S.; Wada, Y. Human Water Use Impacts on the Strength of the Continental Sink for Atmospheric Water. {\em Geophys. Res. Lett.} {\bf 2018}, 45 (9), 4068–4076. https://doi.org/10.1029/2018GL077621.

\bibitem{Keyes13} 
Keyes, D.E., McInnes, L.C., Woodward, C., Gropp, W., Myra, E., Pernice, M., Bell, J., Brown, J., Clo, A., Connors, J., Constantinescu, E., Estep, D., Evans, K., Farhat, C., Hakim, A., Hammond, G., Hansen, G., Hill, J., Isaac, T., et al. (2013). Multiphysics simulations: Challenges and opportunities. {\em International Journal of High Performance Computing Applications} {\bf 27} 4--83, doi:10.1177/1094342012468181.


\bibitem{Koch2016}
J.~Koch, T.~Cornelissen, Z.~Fang, H.~Bogen, B.~H.~Diekkr{\"u}ger, S.~Kollet,
and S.~Stisen (2016).
\newblock Inter-comparison of three distributed hydrological models with
  respect to seasonal variability of soil moisture patterns at a small forested
  catchment.
\newblock {\em J. of Hydrology,}, (533):234--246. https://doi.org/10.1016/j.jhydrol.2015.12.002.


\bibitem{KRM10}
Kollat, J.B., Reed, P.M. and Maxwell, R.M. (2011). Many-objective groundwater monitoring network design using bias-aware ensemble Kalman filtering, evolutionary optimization, and visual analytics. {\em Water Resources Research},doi:10.1029/2010WR009194.

\bibitem{K09} 
Kollet, S.J. (2009). Influence of soil heterogeneity on evapotranspiration under shallow water table conditions: transient, stochastic simulations. {\em Environmental Research Letters} {\bf 4} 1--9, doi:10.1088/1748-9326/4/3/035007.

\bibitem{KCSMMB09}
Kollet, S.J., Cvijanovic, I., Sch{\a"u}ttemeyer, D., Maxwell, R.M., Moene, A.F. and Bayer P (2009). The influence of rain sensible heat, subsurface heat convection and the lower temperature boundary condition on the energy balance at the land surface. {\em Vadose Zone Journal}, doi:10.2136/vzj2009.0005.

\bibitem{KM06}
Kollet, S.~J. and Maxwell, R.~M. (2006). Integrated
surface-groundwater flow
  modeling: A free-surface overland flow boundary condition in a parallel
  groundwater flow model. {\em Advances in Water Resources}, {\bf 29}:945--958 .

\bibitem{KM08a}
Kollet, S.J. and Maxwell, R.M. (2008). Capturing the influence of groundwater dynamics on land surface processes using an integrated, distributed watershed model, { \em Water Resources Research},{\bf 44}: W02402.

\bibitem{KM08b}
Kollet, S.J. and Maxwell, R.M. (2008). Demonstrating fractal scaling of baseflow residence time distributions using a fully-coupled groundwater and land surface model. {\em Geophysical Research Letters}, {\bf 35}, L07402. 

\bibitem{KMWSVVS10}
Kollet, S.J., Maxwell, R.M., Woodward, C.S., Smith, S.G., Vanderborght, J., Vereecken, H., and Simmer, C. (2010). Proof-of-concept of regional scale hydrologic simulations at hydrologic resolution utilizing massively parallel computer resources. {\em Water Resources Research}, 46, W04201, doi:10.1029/2009WR008730.

\bibitem{Kollet2015}
	S.J. Kollet (2015).
\newblock Optimality and inference in hydrology from entropy production
  considerations: synthetic hillslope numerical experiments.
\newblock {\em Hydrol. Earth Syst. Sci. Discuss.}, (12):5123--5149.

\bibitem{Kollet2016} 
Kollet, S. J. (2016). Technical Note: Inference in Hydrology from Entropy Balance Considerations. {\em Hydrol. Earth Syst. Sci.} {\bf 20 (7)}, 2801–2809. https://doi.org/10.5194/hess-20-2801-2016.


\bibitem{Kollet2017}
Kollet, S. J., Sulis, M., Maxwell, R. M., Paniconi, C., Putti, M., Bertoldi, G., … Sudicky, E. (2017). The integrated hydrologic model intercomparison project, IH-MIP2: A second set of benchmark results to diagnose integrated hydrology and feedbacks. {\em Water Resources Research.} https://doi.org/10.1002/2016WR019191.


\bibitem{Kollet2018}
Kollet, S., Gasper, F., Brdar, S., Goergen, K., Hendricks-Franssen, H. J., Keune, J., … Sulis, M. (2018). Introduction of an experimental terrestrial forecasting/monitoring system at regional to continental scales based on the terrestrial systems modeling platform (v1.1.0). {\em Water (Switzerland)}, {\bf 10(11)}. https://doi.org/10.3390/w10111697.


\bibitem{Kuffour2019}
Kuffour, B. N. O. ., Engdahl, N. B. ., Woodward, C. S. ., Condon, L. E. ., Kollet, S., \& Maxwell, R. M. (2019). Simulating Coupled Surface-Subsurface Flows with ParFlow v3.5.0: Capabilities, applications, and ongoing development of an open-source, massively parallel, integrated hydrologic model. {\em Geoscientific Model Development}, (August). https://doi.org/10.5194/gmd-2019-190.


\bibitem{Kurtz2016}
Kurtz, W., He, G., Kollet, S. J., Maxwell, R. M., Vereecken, H., \& Franssen, H. J. H. (2016). TerrSysMP-PDAF (version 1.0): A modular high-performance data assimilation framework for an integrated land surface-subsurface model. {\em Geoscientific Model Development.} https://doi.org/10.5194/gmd-9-1341-2016.


\bibitem{Lim2017}
Lim, T. C., \& Welty, C. (2017). Effects of spatial configuration of imperviousness and green infrastructure networks on hydrologic response in a residential sewershed. {\em Water Resources Research}, {\bf 53(9)}, 8084–8104. https://doi.org/10.1002/2017WR020631.


\bibitem{Lim2018}
Lim, T. C., \& Welty, C. (2018). Assessing variability and uncertainty in green infrastructure planning using a high-resolution surface-subsurface hydrological model and site-monitored flow data. {\em Frontiers in Built Environment,} {\bf 4} (December), 1–15. https://doi.org/10.3389/fbuil.2018.00071.


\bibitem{Lopez2016}
Lopez, S. R., \& Maxwell, R. M. (2016). Identifying Urban Features from LiDAR for a High-Resolution Urban Hydrologic Model. {\em Journal of the American Water Resources Association}, {\bf 52(3)}, 756–768. https://doi.org/10.1111/1752-1688.12425.


\bibitem{Maina2019}
Maina, F. Z., \& Siirila‐Woodburn, E. R. (2019). Watersheds dynamics following wildfires: Nonlinear feedbacks and implications on hydrologic responses. {\em Hydrological Processes}, (August), 1–18. https://doi.org/10.1002/hyp.13568.


\bibitem{Major11} 
Major, E., Benson, D.A., Revielle, J., Ibrahim, H., Dean, A., Maxwell, R.M., Poeter, E. and Dogan, M. (2011). Comparison of Fickian and temporally nonlocal transport theories over many scales in an exhaustively sampled sandstone slab. {\em Water Resources Research} {\bf 47} 1-14, doi:10.1029/2011WR010857.

\bibitem{Markovich2016}
Markovich, K. H., Maxwell, R. M., \& Fogg, G. E. (2016). Hydrogeological response to climate change in alpine hillslopes. {\em Hydrological Processes}, {\bf 30(18)}, 3126–3138. https://doi.org/10.1002/hyp.10851.


\bibitem{MCT00}
Maxwell, R.M., Carle, S.F and Tompson, A.F.B. (2000). Risk-Based Management of Contaminated Groundwater: The Role of Geologic Heterogeneity, Exposure and Cancer Risk in Determining the Performance of Aquifer Remediation, In {\em Proceedings of Computational Methods in Water Resources XII}, Balkema, 533--539.

\bibitem{MWT03} Maxwell, R.M., Welty,C. and Tompson, A.F.B. (2003).
Streamline-based simulation of virus transport resulting from long term
artificial recharge in a heterogeneous aquifer {\em Advances in Water
Resources}, {\bf 22}(3):203--221.

\bibitem{MM05}
Maxwell, R.M. and Miller, N.L. (2005). Development of a coupled land surface and groundwater model.  {\em Journal of Hydrometeorology}, {\bf 6}(3):233--247.

\bibitem{MCK07}
Maxwell, R.M., Chow, F.K. and Kollet, S.J. (2007). The groundwater-land-surface-atmosphere connection: soil moisture effects on the atmospheric boundary layer in fully-coupled simulations. {\em Advances in Water Resources}, {\bf 30}(12):2447--2466.

\bibitem{MWH07} 
Maxwell, R.M., Welty, C. and Harvey, R.W. (2007). Revisiting the Cape Cod Bacteria Injection Experiment Using a Stochastic Modeling Approach, {\em Environmental Science and Technology}, { \bf 41}(15):5548--5558.

\bibitem {MCT08}
Maxwell, R.M., Carle, S.F. and Tompson, A.F.B. (2008).
Contamination, risk, and heterogeneity: on the effectiveness of aquifer remediation. {\em Environmental Geology}, {\bf 54}:1771--1786.

\bibitem{MK08a}
Maxwell, R.M. and Kollet, S.J. (2008). Quantifying the effects of three-dimensional subsurface heterogeneity on Hortonian runoff processes using a coupled numerical, stochastic approach. {\em Advances in Water Resources} {\bf 31}(5): 807--817. 

\bibitem{MK08b}
Maxwell, R.M. and Kollet, S.J. (2008) Interdependence of groundwater dynamics and land-energy feedbacks under climate change. {\em Nature Geoscience} {\bf 1}(10): 665--669.

\bibitem{MTK09}
Maxwell, R.M., Tompson, A.F.B. and Kollet, S.J. (2009) A serendipitous, long-term infiltration experiment: Water and tritium circulation beneath the CAMBRIC trench at the Nevada Test Site. {\em Journal of Contaminant Hydrology} 108(1-2) 12-28, doi:10.1016/j.jconhyd.2009.05.002.

\bibitem{M10}
Maxwell, R.M. (2010). Infiltration in arid environments: Spatial patterns between subsurface heterogeneity and water-energy balances, {\em Vadose Zone Journal} 9, 970--983, doi:10.2136/vzj2010.0014.

\bibitem{MLMSWT10}
Maxwell, R.M., Lundquist, J.K., Mirocha, J.D., Smith, S.G., Woodward, C.S. and Tompson, A.F.B. (2011). Development of a coupled groundwater-atmospheric model. {\em Monthly Weather Review} doi:10.1175/2010MWR3392.

\bibitem{M13}
Maxwell, R.M. (2013). A terrain-following grid transform and preconditioner for parallel, large-scale, integrated hydrologic modeling. {\em Advances in Water Resources} {\bf 53} 109--117, doi:10.1016/j.advwatres.2012.10.001. 

\bibitem{M14} 
Maxwell, R.M., Putti, M., Meyerhoff, S., Delfs, J.-O., Ferguson, I.M., Ivanov, V., Kim, J., Kolditz, O., Kollet, S.J., Kumar, M., Lopez, S., Niu, J., Paniconi, C., Park, Y.-J., Phanikumar, M.S., Shen, C., Sudicky, E. a. and Sulis, M. (2014). Surface-subsurface model intercomparison: A first set of benchmark results to diagnose integrated hydrology and feedbacks. {\em Water Resources Research} {\bf 50} 1531¿1549, doi:10.1002/2013WR013725.

\bibitem{Maxwell2015}
	Maxwell,R.M., L.~E. Condon, and S.~J. Kollet (2015).
\newblock A high-resolution simulation of groundwater and surface water over
  most of the continental us with the integrated hydrologic model parflow v3.
\newblock {\em Geoscientific Model Development}, 8(3):923--937.

\bibitem{Maxwell2016}
Maxwell,R.M., L.~E. Condon, S.~J. Kollet, K. Maher, R. Haggerty, and
M.~M. Forrester (2016).
\newblock The imprint of climate and geology on the residence times of
  groundwater.
\newblock {\em Geophysical Research Letters}, 43(2):701--708.
\newblock 2015GL066916.


\bibitem{Maxwell2016b}
Maxwell, R. M., \& Condon, L. E. (2016). Connections between groundwater flow and transpiration partitioning. {\em Science}, {\bf 353(6297)}, 377–380. https://doi.org/10.1126/science.aaf7891.


\bibitem{Maxwell2019}
Maxwell, R. M., Condon, L. E., Danesh-Yazdi, M., \& Bearup, L. A. (2019). Exploring source water mixing and transient residence time distributions of outflow and evapotranspiration with an integrated hydrologic model and Lagrangian particle tracking approach. {\em Ecohydrology}, {\bf 12(1)}, 1–10. https://doi.org/10.1002/eco.2042.

\bibitem{M14} 
Maxwell, R.M., Putti, M., Meyerhoff, S., Delfs, J.-O., Ferguson, I.M., Ivanov, V., Kim, J., Kolditz, O., Kollet, S.J., Kumar, M., Lopez, S., Niu, J., Paniconi, C., Park, Y.-J., Phanikumar, M.S., Shen, C., Sudicky, E. a. and Sulis, M. (2014). Surface-subsurface model intercomparison: A first set of benchmark results to diagnose integrated hydrology and feedbacks. {\em Water Resources Research} {\bf 50} 1531¿1549, doi:10.1002/2013WR013725.

\bibitem{MTK09}
Maxwell, R.M., Tompson, A.F.B. and Kollet, S.J. (2009) A serendipitous, long-term infiltration experiment: Water and tritium circulation beneath the CAMBRIC trench at the Nevada Test Site. {\em Journal of Contaminant Hydrology} 108(1-2) 12-28, doi:10.1016/j.jconhyd.2009.05.002.

\bibitem{MWH07} 
Maxwell, R.M., Welty, C. and Harvey, R.W. (2007). Revisiting the Cape Cod Bacteria Injection Experiment Using a Stochastic Modeling Approach, {\em Environmental Science and Technology}, { \bf 41}(15):5548--5558.

\bibitem{MWT03} Maxwell, R.M., Welty,C. and Tompson, A.F.B. (2003).
Streamline-based simulation of virus transport resulting from long term
artificial recharge in a heterogeneous aquifer {\em Advances in Water
Resources}, {\bf 22}(3):203--221.

\bibitem{Meyerhoff11} 
Meyerhoff, S.B. and Maxwell, R.M. (2011). Quantifying the effects of subsurface heterogeneity on hillslope runoff using a stochastic approach. {\em Hydrogeology Journal} {\bf 19} 1515¿1530, doi:10.1007/s10040-011-0753-y.

\bibitem{Meyerhoff14a} 
Meyerhoff, S.B., Maxwell, R.M., Graham, W.D. and Williams, J.L. (2014). Improved hydrograph prediction through subsurface characterization: conditional stochastic hillslope simulations. {\em Hydrogeology Journal} doi:10.1007/s10040-014-1112-6.

\bibitem{Meyerhoff14b} 
Meyerhoff, S.B., Maxwell, R.M., Revil, A., Martin, J.B., Karaoulis, M. and Graham, W.D. (2014). Characterization of groundwater and surface water mixing in a semiconfined karst aquifer using time-lapse electrical resistivity tomography. {\em Water Resources Research} {\bf 50} 2566¿2585, doi:10.1002/2013WR013991.

\bibitem{Mikkelson13} 
Mikkelson, K.M., Maxwell, R.M., Ferguson, I., Stednick, J.D., McCray, J.E. and Sharp, J.O. (2013). Mountain pine beetle infestation impacts: modeling water and energy budgets at the hill-slope scale. {\em Ecohydrology} {\bf 6} doi:10.1002/eco.278.

\bibitem{Moqbel2018}
Moqbel, S.; Abu-El-Sha’r, W. (2018). Modeling Groundwater Flow and Solute Transport at Azraq Basin Using Parflow and Slim-Fast. {\em Jordan J. Civ. Eng.} {\bf 12 (2)}, 263–278.

\bibitem{Penn2016}
Penn, C. A., Bearup, L. A., Maxwell, R. M., \& Clow, D. W. (2016). Numerical experiments to explain multiscale hydrological responses to mountain pine beetle tree mortality in a headwater watershed. {\em Water Resources Research}, {\bf 52}, 3143–3161. doi:10.1002/ 2015WR018300.


\bibitem{Pribulick2016}
Pribulick, C. E., Foster, L. M., Bearup, L. A., Navarre-Sitchler, A. K., Williams, K. H., Carroll, R. W. H., \& Maxwell, R. M. (2016). Contrasting the hydrologic response due to land cover and climate change in a mountain headwaters system. {\em Ecohydrology}, {\bf 9(8)}, 1431–1438. https://doi.org/10.1002/eco.1779.


\bibitem{Rahman2015}
Rahman, M., M.~Sulis, and S.J. Kollet (2015).
\newblock Evaluating the dual-boundary forcing concept in subsurface-land
  surface interactions of the hydrological cycle.
\newblock {\em Hydrological Processes}.

\bibitem{Rahman2015a}
Rahman,M. M.~Sulis, and S.J. Kollet (2015).
\newblock The subsurface-land surface-atmosphere connection under convective
  conditions.
\newblock {\em Advances in Water Resour.}, (83):240--249.

\bibitem{Rahman2016}
Rahman, M., Sulis, M., \& Kollet, S. J. (2016). Evaluating the dual-boundary forcing concept in subsurface-land surface interactions of the hydrological cycle. {\em Hydrological Processes}, {\bf 30(10)}, 1563–1573. https://doi.org/10.1002/hyp.10702.


\bibitem{Rahman2018}
Rahman, M., Rosolem, R., Kollet, S. J., \& Wagener, T. (2018). Towards a computationally efficient free-surface groundwater flow boundary condition for large-scale hydrological modelling. {\em Advances in Water Resources}, {\bf 123} (December 2018), 225–233. https://doi.org/10.1016/j.advwatres.2018.11.015.

\bibitem{Reyes2015}
Reyes,R., R.M. Maxwell, and T.~S. Hogue (2015).
\newblock Impact of lateral flow and spatial scaling on the simulation of
  semi-arid urban land surfaces in an integrated hydrologic and land surface
  model.
\newblock {\em Hydrological Processes}.

\bibitem{Reyes2018}
Reyes, B., Hogue, T., \& Maxwell, R. (2018). Urban irrigation suppresses land surface temperature and changes the hydrologic regime in semi-arid regions. {\em Water (Switzerland)}, {\bf 10(11)}. https://doi.org/10.3390/w10111563.


\bibitem{RMC10}
Rihani, J., Maxwell, R.M., Chow, F.K. (2010). Coupling groundwater and land-surface processes: Idealized simulations to identify effects of terrain and subsurface heterogeneity on land surface energy fluxes. {\em Water Resources Research} 46, W12523, doi:10.1029/2010WR009111.


\bibitem{Rihani2015}
Rihani, J.F., F.~K. Chow, and R.~M. Maxwell (2015).
\newblock Isolating effects of terrain and soil moisture heterogeneity on the
  atmospheric boundary layer: Idealized simulations to diagnose land-atmosphere
  feedbacks.
\newblock {\em Journal of Advances in Modeling Earth Systems}, 7(2):915--937.

\bibitem{Schalge2019}
Schalge, B., Haefliger, V., Kollet, S., \& Simmer, C. (2019). Improvement of surface run-off in the hydrological model ParFlow by a scale-consistent river parameterization. {\em Hydrological Processes}, (October 2017), 2006–2019. https://doi.org/10.1002/hyp.13448.

\bibitem{Seck2015}
Seck,A. C. Welty, and R.~M. Maxwell (2015).
\newblock Spin-up behavior and effects of initial conditions for an integrated
  hydrologic model.
\newblock {\em Water Resources Research}, 51(4):2188--2210.


\bibitem{Shrestha14} 
Shrestha, P., Sulis, M., Masbou, M., Kollet, S. and Simmer, C. (2014). A scale-consistent Terrestrial Systems Modeling Platform based on COSMO, CLM and ParFlow. {\em Monthly Weather Review} doi:10.1175/MWR-D-14-00029.1.

\bibitem{Shrestha2015}
Shrestha,P., M. Sulis, C.~Simmer, and S.~Kollet (2015).
\newblock Impacts of grid resolution on surface energy fluxes simulated with an
  integrated surface-groundwater flow model.
\newblock {\em Hydrol. Earth Syst. Sci.}, 19:4317--4326.

\bibitem{Shrestha2018}
Shrestha, P., Sulis, M., Simmer, C., \& Kollet, S. (2018). Effects of horizontal grid resolution on evapotranspiration partitioning using TerrSysMP. {\em Journal of Hydrology}, {\bf 557}, 910–915. https://doi.org/10.1016/j.jhydrol.2018.01.024.

\bibitem{SNSMM10}
Siirila, E.R., Navarre-Sitchler, A.K., Maxwell, R.M. and McCray, J.E. (2012). A quantitative methodology to assess the risks to human health from CO2 leakage into groundwater. {\em Advances in Water Resources}, {\bf 36}, 146-164, doi:10.1016/j.advwatres.2010.11.005.

\bibitem{Siirila12a} 
Siirila, E.R. and Maxwell, R.M. (2012). A new perspective on human health risk assessment: Development of a time dependent methodology and the effect of varying exposure durations. {\em Science of The Total Environment} {\bf 431} 221-232, doi:10.1016/j.scitotenv.2012.05.030.

\bibitem{Siirila12b} 
Siirila, E.R. and Maxwell, R.M. (2012). Evaluating effective reaction rates of kinetically driven solutes in large-scale, statistically anisotropic media: Human health risk implications. {\em Water Resources Research} {\bf 48} 1-23, doi:10.1029/2011WR011516.

\bibitem{Woodburn2018}
Siirila-Woodburn, E. R., Steefel, C. I., Williams, K. H., \& Birkholzer, J. T. (2018). Predicting the impact of land management decisions on overland flow generation: Implications for cesium migration in forested Fukushima watersheds. {\em Advances in Water Resources}, {\bf 113} (January), 42–54. https://doi.org/10.1016/j.advwatres.2018.01.008.

\bibitem{Srivastava2014}
Srivastava,V., W. Graham, R. Muñoz-Carpena, and R.~M. Maxwell (2014).
\newblock Insights on geologic and vegetative controls over hydrologic behavior
  of a large complex basin--global sensitivity analysis of an integrated
  parallel hydrologic model.
\newblock {\em Journal of Hydrology}, 519, Part B:2238 -- 2257.


\bibitem{SMPMPK10}
Sulis, M., Meyerhoff, S., Paniconi, C., Maxwell, R.M., Putti, M. and Kollet, S.J. (2010). A comparison of two physics-based numerical models for simulating surface water-groundwater interactions. { \em Advances in Water Resources}, 33(4), 456-467, doi:10.1016/j.advwatres.2010.01.010.

\bibitem{Sulis2017}
Sulis, M., Williams, J. L., Shrestha, P., Diederich, M., Simmer, C., Kollet, S. J., \& Maxwell, R. M. (2017). Coupling Groundwater, Vegetation, and Atmospheric Processes: A Comparison of Two Integrated Models. {\em Journal of Hydrometeorology}. https://doi.org/10.1175/JHM-D-16-0159.1.


\bibitem{Sulis2018}
Sulis, M., Keune, J., Shrestha, P., Simmer, C., \& Kollet, S. J. (2018). Quantifying the Impact of Subsurface-Land Surface Physical Processes on the Predictive Skill of Subseasonal Mesoscale Atmospheric Simulations. {\em Journal of Geophysical Research: Atmospheres}, {\bf 123(17)}, 9131–9151. https://doi.org/10.1029/2017JD028187.

\bibitem{Sweetenham2017}
Sweetenham, M. G., Maxwell, R. M., \& Santi, P. M. (2017). Assessing the timing and magnitude of precipitation-induced seepage into tunnels bored through fractured rock. {\em Tunnelling and Underground Space Technology}, {\bf 65}, 62–75. http://dx.doi.org/10.1016/j.tust.2017.02.003.

\bibitem{TAG89}
Tompson, A.F.B., Ababou, R. and Gelhar, L.W. (1989).
\newblock Implementation of of the three-dimensional turning bands random field
  generator.
\newblock {\em Water Resources Research}, {\bf 25}(10):2227--2243.

\bibitem{TFSBA98}  Tompson, A.F.B., Falgout, R.D., Smith, S.G., Bosl, W.J. and
Ashby, S.F. (1998). Analysis of subsurface contaminant migration and
remediation using high performance computing. {\em Advances in Water
Resources}, { \bf 22}(3):203--221.

\bibitem{TBP99}
Tompson, A. F. B., Bruton, C. J. and Pawloski,  G. A. eds. (1999b). {\em Evaluation of the hydrologic source term from underground nuclear tests in Frenchman Flat at the Nevada Test Site: The CAMBRIC test}, Lawrence Livermore National Laboratory, Livermore, CA (UCRL-ID-132300), 360pp. 

\bibitem{TCRM99}  Tompson, A.F.B., Carle, S.F., Rosenberg, N.D. and Maxwell, R.M. 
 (1999). Analysis of groundwater migration from artificial recharge in a large
 urban aquifer: A simulation perspective, {\em Water Resources
Research}, {\bf 35}(10):2981--2998.

\bibitem{Teal02}
Tompson AFB., Bruton, C.J., Pawloski, G.A., Smith, D.K., Bourcier, W.L., Shumaker, D.E., Kersting, A.B., Carle, S.F. and Maxwell, R.M. (2002). On the evaluation of groundwater contamination from underground nuclear tests.  {\em Environmental Geology}, {\bf 42}(2-3):235--247.

\bibitem{TMCZPS05}
Tompson, A. F. B., Maxwell, R. M., Carle, S. F., Zavarin, M., Pawloski, G. A. and Shumaker, D. E. (2005). {\em Evaluation of the Non-Transient Hydrologic Source Term from the CAMBRIC Underground Nuclear Test in Frenchman Flat, Nevada Test Site}, Lawrence Livermore National Laboratory, Livermore, CA, UCRL-TR-217191.

\bibitem{VanGenuchten80}
{van Genuchten}, M.Th.(1980). 
\newblock A closed form equation for predicting the hydraulic conductivity of
  unsaturated soils.
\newblock {\em Soil Sci. Soc. Am. J.}, {\bf 44}:892--898.

\bibitem{welch.95}
Welch, B. (1995)
\newblock {\em Practical Programming in {T}{C}{L} and {T}{K}}.
\newblock Prentice Hall.

\bibitem{Woodward98}
Woodward, C.S. (1998),
\newblock A {N}ewton-{K}rylov-{M}ultigrid solver for variably saturated flow
  problems.
\newblock In {\em Proceedings of the XIIth International Conference on
  Computational Methods in Water Resources}, June.

\bibitem{WGM02}
Woodward, C.S., Grant,  K.E., and Maxwell, R.M. (2002). Applications of Sensitivity Analysis to Uncertainty Quantification for Variably Saturated Flow.
\newblock In {\em Proceedings of the XIVth International Conference on Computational Methods in Water Resources, Amsterdam}, The Netherlands, June.

\bibitem{Williams11} 
Williams, J.L. and Maxwell, R.M. (2011). Propagating Subsurface Uncertainty to the Atmosphere Using Fully Coupled Stochastic Simulations. {\em Journal of Hydrometeorology} {\bf 12} 690-701, doi:10.1175/2011JHM1363.1.

\bibitem{Williams13} 
Williams, J.L., Maxwell, R.M. and Monache, L.D. (2013). Development and verification of a new wind speed forecasting system using an ensemble Kalman filter data assimilation technique in a fully coupled hydrologic and atmospheric model. {\em Journal of Advances in Modeling Earth Systems} {\bf 5} 785-800, doi:10.1002/jame.20051.

\bibitem{Zhang2018}
Zhang, H., Kurtz, W., Kollet, S., Vereecken, H., \& Franssen, H. J. H. (2018). Comparison of different assimilation methodologies of groundwater levels to improve predictions of root zone soil moisture with an integrated terrestrial system model. {\em Advances in Water Resources}, {\bf 111}(May), 224–238. https://doi.org/10.1016/j.advwatres.2017.11.003.


\bibitem{Zipper2019}
Zipper, S. C., Keune, J., \& Kollet, S. J. (2019). Land use change impacts on European heat and drought: Remote land-atmosphere feedbacks mitigated locally by shallow groundwater. {\em Environmental Research Letters}, {\em 14(4)}. https://doi.org/10.1088/1748-9326/ab0db3.

\bibitem{endian}
{\em Endianness}, Wikipedia Entry: http://en.wikipedia.org/wiki/Endianness



\end{thebibliography}
