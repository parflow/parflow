%=============================================================================
% Reference Section: Grid Reference
%=============================================================================

\newpage
\section{Grid Reference}
\label{Grid Reference}

%=============================================================================
% Reference: Grid Structures
%=============================================================================

\subsection{Grid Structures}
\label{Grid Structures}

Many of the components in the following structures relate intuitively
to the discussion in \ref{Gridding}.
We will discuss here only those components which need explanation.

\begin{display}\begin{verbatim}
typedef struct
{
   double   X,  Y,  Z;    /* Bottom-lower-left corner in real-space */
   int      NX, NY, NZ;   /* Size in each coordinate direction */
   double   DX, DY, DZ;   /* Spacing in each coordinate direction */

} Background;
\end{verbatim}\end{display}

There is only one \code{Background} structure currently in \parflow{}.
It is currently global.

\begin{display}\begin{verbatim}
typedef struct
{
   int  ix, iy, iz;      /* Bottom-lower-left corner in index-space */
   int  nx, ny, nz;      /* Size */
   int  sx, sy, sz;      /* Striding factors */
   int  rx, ry, rz;      /* Refinement over the background grid */
   int  level;           /* Refinement level = rx + ry + rz */

   int  process;         /* Process containing this subgrid */

} Subregion;
\end{verbatim}\end{display}

The \code{process} component of the \code{Subregion} structure is
intended to associate a subregion with a particular process.
However, it is often times not used, and may be replaced in the
future by some other mechanism for achieving this task.

\begin{display}\begin{verbatim}
typedef struct
{
   Subregion  **subregions;   /* Array of pointers to subregions */
   int          size;         /* Size of subregion array */

} SubregionArray;
\end{verbatim}\end{display}

\begin{display}\begin{verbatim}
typedef struct
{
   SubregionArray  **subregion_arrays;   /* Array of pointers to
                                          * subregion arrays */
   int               size;               /* Size of region */

} Region;
\end{verbatim}\end{display}

\begin{display}\begin{verbatim}
typedef Subregion  Subgrid;
\end{verbatim}\end{display}

\begin{display}\begin{verbatim}
typedef SubregionArray  SubgridArray;
\end{verbatim}\end{display}

\begin{display}\begin{verbatim}
typedef struct
{
   SubgridArray  *subgrids;     /* Array of subgrids in this process */

   SubgridArray  *all_subgrids; /* Array of all subgrids in the grid */

   int            size;         /* Total number of grid points */

   ComputePkg   **compute_pkgs;

} Grid;
\end{verbatim}\end{display}

The \code{subgrids} component of the \code{Grid} structure above, is a
subgrid-array of subgrids associated with my process.
The \code{all_subgrids} component is a subgrid-array of all subgrids
in the grid.
It is important to note that \code{subgrids} contains pointers into
\code{all_subgrids}.
The \code{compute_pkgs} component is an array of pointers to
\code{ComputePkg} structures.
This structure is related to communications within \parflow{} and
really should not appear in the \code{Grid} structures.
Hopefully, this we be fixed at some point in the future.

%=============================================================================
% Reference: Grid Accessors
%=============================================================================

\subsection{Grid Accessors}
\label{Grid Accessors}

Ignore the format of this section; it is currently in debate.

\begin{display}\begin{verbatim}
Background      *background;
Subregion       *subregion;
SubregionArray  *subregion_array;
Region          *region;
Subgrid         *subgrid;
SubgridArray    *subgrid_array;
Grid            *grid;
int              i;
\end{verbatim}\end{display}

\begin{display}\begin{verbatim}

BackgroundX(background)   background -> X
BackgroundY(background)   background -> Y
BackgroundZ(background)   background -> Z
BackgroundNX(background)  background -> NX
BackgroundNY(background)  background -> NY
BackgroundNZ(background)  background -> NZ
BackgroundDX(background)  background -> DX
BackgroundDY(background)  background -> DY
BackgroundDZ(background)  background -> DZ

\end{verbatim}\end{display}

\begin{display}\begin{verbatim}

SubregionIX(subregion)      subregion -> ix
SubregionIY(subregion)      subregion -> iy
SubregionIZ(subregion)      subregion -> iz
SubregionNX(subregion)      subregion -> nx
SubregionNY(subregion)      subregion -> ny
SubregionNZ(subregion)      subregion -> nz
SubregionSX(subregion)      subregion -> sx
SubregionSY(subregion)      subregion -> sy
SubregionSZ(subregion)      subregion -> sz
SubregionRX(subregion)      subregion -> rx
SubregionRY(subregion)      subregion -> ry
SubregionRZ(subregion)      subregion -> rz
SubregionLevel(subregion)   subregion -> level
SubregionProcess(subregion) subregion -> process

\end{verbatim}\end{display}

\begin{display}\begin{verbatim}

SubregionArraySubregion(subregion_array, i)  subregion_array -> subregions[i]
SubregionArraySize(subregion_array)          subregion_array -> size

\end{verbatim}\end{display}

\begin{display}\begin{verbatim}

RegionSubregionArray(region, i)  region -> subregion_arrays[i]
RegionSize(region)               region -> size

\end{verbatim}\end{display}

\begin{display}\begin{verbatim}

SubgridIX(subgrid)       SubregionIX(subgrid)
SubgridIY(subgrid)       SubregionIY(subgrid)
SubgridIZ(subgrid)       SubregionIZ(subgrid)
SubgridNX(subgrid)       SubregionNX(subgrid)
SubgridNY(subgrid)       SubregionNY(subgrid)
SubgridNZ(subgrid)       SubregionNZ(subgrid)
SubgridRX(subgrid)       SubregionRX(subgrid)
SubgridRY(subgrid)       SubregionRY(subgrid)
SubgridRZ(subgrid)       SubregionRZ(subgrid)
SubgridLevel(subgrid)    SubregionLevel(subgrid)
SubgridProcess(subgrid)  SubregionProcess(subgrid)

\end{verbatim}\end{display}

\begin{display}\begin{verbatim}

SubgridArraySubgrid(subgrid_array, i)  SubregionArraySubregion(subgrid_array, i)
SubgridArraySize(subgrid_array)        SubregionArraySize(subgrid_array)

\end{verbatim}\end{display}

\begin{display}\begin{verbatim}

GridSubgrids(grid)       grid -> subgrids
GridAllSubgrids(grid)    grid -> all_subgrids
GridSize(grid)           grid -> size
GridComputePkgs(grid)    grid -> compute_pkgs
GridComputePkg(grid, i)  grid -> compute_pkgs[i]
GridSubgrid(grid, i)     SubgridArraySubgrid(GridSubgrids(grid), i)
GridNumSubgrids(grid)    SubgridArraySize(GridSubgrids(grid))

\end{verbatim}\end{display}

%=============================================================================
% Reference: ReadBackground
%=============================================================================

\newpage
\subsection{ReadBackground}
\label{ReadBackground}

\index{\findex{ReadBackground}}

\begin{deftypefn}{Function}{Background *}{ReadBackground}(amps_File \var{file})

%=========================== DESCRIPTION
\DESCRIPTION
Reads in a background description from \var{file}, then
creates and returns a new \code{Background} structure.

%=========================== SEE ALSO
\SEEALSO
\vref{Grid Structures}{Grid Structures}\\
\vref{Gridding}{Gridding}

\end{deftypefn}

%=============================================================================
% Reference: NewSubregion
%=============================================================================

\newpage
\subsection{NewSubregion}
\label{NewSubregion}

\index{\findex{NewSubregion}}

\begin{deftypefn}{Function}{Subregion *}{NewSubregion}(int \var{ix}, int \var{iy}, int \var{iz}, int \var{nx}, int \var{ny}, int \var{nz}, int \var{sx}, int \var{sy}, int \var{sz}, int \var{rx}, int \var{ry}, int \var{rz}, int \var{process})

%=========================== DESCRIPTION
\DESCRIPTION
Creates and returns a pointer to a new \code{Subregion} structure.

%=========================== SEE ALSO
\SEEALSO
\vref{Grid Structures}{Grid Structures}\\
\vref{Gridding}{Gridding}

\end{deftypefn}

%=============================================================================
% Reference: NewSubregionArray
%=============================================================================

\newpage
\subsection{NewSubregionArray}
\label{NewSubregionArray}

\index{\findex{NewSubregionArray}}

\begin{deftypefn}{Function}{SubregionArray *}{NewSubregionArray}()

%=========================== DESCRIPTION
\DESCRIPTION
Creates and returns a pointer to a new \code{SubregionArray} structure
of size 0.

%=========================== SEE ALSO
\SEEALSO
\vref{Grid Structures}{Grid Structures}\\
\vref{AppendSubregion}{AppendSubregion}\\
\vref{AppendSubregionArray}{AppendSubregionArray}\\
\vref{Gridding}{Gridding}

\end{deftypefn}

%=============================================================================
% Reference: NewRegion
%=============================================================================

\newpage
\subsection{NewRegion}
\label{NewRegion}

\index{\findex{NewRegion}}

\begin{deftypefn}{Function}{Region *}{NewRegion}(int \var{size})

%=========================== DESCRIPTION
\DESCRIPTION
Creates and returns a pointer to a new \code{Region} structure
containing \var{size} pointers to new \code{SubregionArray} structures.

%=========================== SEE ALSO
\SEEALSO
\vref{Grid Structures}{Grid Structures}\\
\vref{NewSubregionArray}{NewSubregionArray}\\
\vref{AppendSubregion}{AppendSubregion}\\
\vref{AppendSubregionArray}{AppendSubregionArray}\\
\vref{Gridding}{Gridding}

\end{deftypefn}

%=============================================================================
% Reference: NewSubgrid
%=============================================================================

\newpage
\subsection{NewSubgrid}
\label{NewSubgrid}

\index{\findex{NewSubgrid}}

\begin{deftypefn}{Function}{Subgrid *}{NewSubgrid}(int \var{ix}, int \var{iy}, int \var{iz}, int \var{nx}, int \var{ny}, int \var{nz}, int \var{rx}, int \var{ry}, int \var{rz}, int \var{process})

%=========================== DESCRIPTION
\DESCRIPTION
Creates and returns a pointer to a new \code{Subgrid} structure.

%=========================== NOTES
\NOTES
This is currently a macro (\ref{NewSubregion}).

%=========================== SEE ALSO
\SEEALSO
\vref{Grid Structures}{Grid Structures}\\
\vref{Gridding}{Gridding}

\end{deftypefn}

%=============================================================================
% Reference: NewSubgridArray
%=============================================================================

\newpage
\subsection{NewSubgridArray}
\label{NewSubgridArray}

\index{\findex{NewSubgridArray}}

\begin{deftypefn}{Function}{SubgridArray *}{NewSubgridArray}()

%=========================== DESCRIPTION
\DESCRIPTION
Creates and returns a pointer to a new \code{SubgridArray} structure
of size 0.

%=========================== SEE ALSO
\SEEALSO
\vref{Grid Structures}{Grid Structures}\\
\vref{AppendSubgrid}{AppendSubgrid}\\
\vref{AppendSubgridArray}{AppendSubgridArray}\\
\vref{Gridding}{Gridding}

\end{deftypefn}

%=============================================================================
% Reference: NewGrid
%=============================================================================

\newpage
\subsection{NewGrid}
\label{NewGrid}

\index{\findex{NewGrid}}

\begin{deftypefn}{Function}{Grid *}{NewGrid}({SubgridArray *} \var{subgrids}, {SubgridArray *} \var{all\_subgrids})

%=========================== DESCRIPTION
\DESCRIPTION
Creates and returns a pointer to a new \code{Grid} structure.
The \code{size} component is computed and the \code{compute_pkgs}
component is set to \code{NULL}.

%=========================== SEE ALSO
\SEEALSO
\vref{Grid Structures}{Grid Structures}\\
\vref{Gridding}{Gridding}

\end{deftypefn}

%=============================================================================
% Reference: FreeBackground
%=============================================================================

\newpage
\subsection{FreeBackground}
\label{FreeBackground}

\index{\findex{FreeBackground}}

\begin{deftypefn}{Function}{void}{FreeBackground}({Background *} \var{background})

%=========================== DESCRIPTION
\DESCRIPTION
Frees a \code{Background} structure.

%=========================== SEE ALSO
\SEEALSO
\vref{Grid Structures}{Grid Structures}\\
\vref{Gridding}{Gridding}

\end{deftypefn}

%=============================================================================
% Reference: FreeSubregion
%=============================================================================

\newpage
\subsection{FreeSubregion}
\label{FreeSubregion}

\index{\findex{FreeSubregion}}

\begin{deftypefn}{Function}{void}{FreeSubregion}({Subregion *} \var{subregion})

%=========================== DESCRIPTION
\DESCRIPTION
Frees a \code{Subregion} structure.

%=========================== SEE ALSO
\SEEALSO
\vref{Grid Structures}{Grid Structures}\\
\vref{Gridding}{Gridding}

\end{deftypefn}

%=============================================================================
% Reference: FreeSubregionArray
%=============================================================================

\newpage
\subsection{FreeSubregionArray}
\label{FreeSubregionArray}

\index{\findex{FreeSubregionArray}}

\begin{deftypefn}{Function}{void}{FreeSubregionArray}({SubregionArray *} \var{subregion\_array})

%=========================== DESCRIPTION
\DESCRIPTION
Frees a \code{SubregionArray} structure, and all \code{Subregion}
structures pointed to by it.
Often times one may want to free up the \code{SubregionArray} structure
without freeing its \code{Subregion} structures.
This is easily achieved as in the following example.

%=========================== EXAMPLES
\EXAMPLE
\mbox{}
\begin{display}\begin{verbatim}
SubregionArray  *subregion_array;

   ...

SubregionArraySize(subregion_array) = 0;
FreeSubregionArray(subregion_array);
\end{verbatim}\end{display}

%=========================== SEE ALSO
\SEEALSO
\vref{Grid Structures}{Grid Structures}\\
\vref{FreeSubregion}{FreeSubregion}\\
\vref{Gridding}{Gridding}

\end{deftypefn}

%=============================================================================
% Reference: FreeRegion
%=============================================================================

\newpage
\subsection{FreeRegion}
\label{FreeRegion}

\index{\findex{FreeRegion}}

\begin{deftypefn}{Function}{void}{FreeRegion}({Region *} \var{region})

%=========================== DESCRIPTION
\DESCRIPTION
Frees a \code{Region} structure and all \code{SubregionArray}
structures pointed to by it.
Sometimes one may want to free up the \code{Region} structure
without freeing its \code{SubregionArray} structures.
This is easily achieved as in the following example.

%=========================== EXAMPLES
\EXAMPLE
\mbox{}
\begin{display}\begin{verbatim}
Region  *region;

   ...

RegionSize(region) = 0;
FreeRegion(region);
\end{verbatim}\end{display}

%=========================== SEE ALSO
\SEEALSO
\vref{Grid Structures}{Grid Structures}\\
\vref{FreeSubregionArray}{FreeSubregionArray}\\
\vref{Gridding}{Gridding}

\end{deftypefn}

%=============================================================================
% Reference: FreeSubgrid
%=============================================================================

\newpage
\subsection{FreeSubgrid}
\label{FreeSubgrid}

\index{\findex{FreeSubgrid}}

\begin{deftypefn}{Function}{void}{FreeSubgrid}({Subgrid *} \var{subgrid})

%=========================== DESCRIPTION
\DESCRIPTION
Frees a \code{Subgrid} structure.

%=========================== NOTES
\NOTES
This is currently a macro (\ref{FreeSubregion}).

%=========================== SEE ALSO
\SEEALSO
\vref{Grid Structures}{Grid Structures}\\
\vref{Gridding}{Gridding}

\end{deftypefn}

%=============================================================================
% Reference: FreeSubgridArray
%=============================================================================

\newpage
\subsection{FreeSubgridArray}
\label{FreeSubgridArray}

\index{\findex{FreeSubgridArray}}

\begin{deftypefn}{Function}{void}{FreeSubgridArray}({SubgridArray *} \var{subgrid\_array})

%=========================== DESCRIPTION
\DESCRIPTION
Frees a \code{SubgridArray} structure, and all \code{Subgrid}
structures pointed to by it.
Often times one may want to free up the \code{SubgridArray} structure
without freeing its \code{Subgrid} structures.
This is easily achieved as in the following example.

%=========================== EXAMPLES
\EXAMPLE
\mbox{}
\begin{display}\begin{verbatim}
SubgridArray  *subgrid_array;

   ...

SubgridArraySize(subgrid_array) = 0;
FreeSubgridArray(subgrid_array);
\end{verbatim}\end{display}

%=========================== NOTES
\NOTES
This is currently a macro (\ref{FreeSubregionArray}).

%=========================== SEE ALSO
\SEEALSO
\vref{Grid Structures}{Grid Structures}\\
\vref{FreeSubgrid}{FreeSubgrid}\\
\vref{Gridding}{Gridding}

\end{deftypefn}

%=============================================================================
% Reference: FreeGrid
%=============================================================================

\newpage
\subsection{FreeGrid}
\label{FreeGrid}

\index{\findex{FreeGrid}}

\begin{deftypefn}{Function}{void}{FreeGrid}({Grid *} \var{grid})

%=========================== DESCRIPTION
\DESCRIPTION
Frees a \code{Grid} structure and all structures contained therein.

%=========================== SEE ALSO
\SEEALSO
\vref{Grid Structures}{Grid Structures}\\
\vref{FreeSubgridArray}{FreeSubgridArray}\\
\vref{Gridding}{Gridding}

\end{deftypefn}

%=============================================================================
% Reference: ForSubregionI
%=============================================================================

\newpage
\subsection{ForSubregionI}
\label{ForSubregionI}

\index{\findex{ForSubregionI}}

\begin{defmac} ForSubregionI (int \var{i}, {SubregionArray *} \var{subregion\_array})

%=========================== DESCRIPTION
\DESCRIPTION
Loops over all subregion indices in \var{subregion\_array}, setting
\var{i} to each index.

%=========================== EXAMPLES
\EXAMPLE
\mbox{}
\begin{display}\begin{verbatim}
SubregionArray  *subregion_array;
Subregion       *subregion;
int              i;

ForSubregionI(i, subregion_array)
{
   subregion = SubregionArraySubregion(subregion_array, i);

   ...
}
\end{verbatim}\end{display}

%=========================== SEE ALSO
\SEEALSO
\vref{Grid Structures}{Grid Structures}\\
\vref{Gridding}{Gridding}

\end{defmac}

%=============================================================================
% Reference: ForSubregionArrayI
%=============================================================================

\newpage
\subsection{ForSubregionArrayI}
\label{ForSubregionArrayI}

\index{\findex{ForSubregionArrayI}}

\begin{defmac} ForSubregionArrayI (int \var{i}, {Region *} \var{region})

%=========================== DESCRIPTION
\DESCRIPTION
Loops over all subregion-array indices in \var{region}, setting
\var{i} to each index.

%=========================== EXAMPLES
\EXAMPLE
\mbox{}
\begin{display}\begin{verbatim}
Region          *region;
SubregionArray  *subregion_array;
int              i;

ForSubregionArrayI(i, region)
{
   subregion_array = RegionSubregionArray(region, i);

   ...
}
\end{verbatim}\end{display}

%=========================== NOTES
%\NOTES

%=========================== SEE ALSO
\SEEALSO
\vref{Grid Structures}{Grid Structures}\\
\vref{Gridding}{Gridding}

\end{defmac}

%=============================================================================
% Reference: ForSubgridI
%=============================================================================

\newpage
\subsection{ForSubgridI}
\label{ForSubgridI}

\index{\findex{ForSubgridI}}

\begin{defmac} ForSubgridI (int \var{i}, {SubgridArray *} \var{subgrid\_array})

%=========================== DESCRIPTION
\DESCRIPTION
Loops over all subgrid indices in \var{subgrid\_array}, setting
\var{i} to each index.

%=========================== EXAMPLES
\EXAMPLE
\mbox{}
\begin{display}\begin{verbatim}
SubgridArray *subgrid_array;
Subgrid      *subgrid;
int           i;

ForSubgridI(i, subgrid_array)
{
   subgrid = SubgridArraySubgrid(subgrid_array, i);

   ...
}
\end{verbatim}\end{display}

%=========================== SEE ALSO
\SEEALSO
\vref{Grid Structures}{Grid Structures}\\
\vref{Gridding}{Gridding}

\end{defmac}

%=============================================================================
% Reference: SubregionXYZ
%=============================================================================

\newpage
\subsection{SubregionXYZ}
\label{SubregionXYZ}

\index{\findex{SubregionXYZ}}

\begin{deftypefn}{Function}{double}{SubregionX}({Subregion *} \var{subregion})
\end{deftypefn}
\begin{deftypefn}{Function}{double}{SubregionY}({Subregion *} \var{subregion})
\end{deftypefn}
\begin{deftypefn}{Function}{double}{SubregionZ}({Subregion *} \var{subregion})

%=========================== DESCRIPTION
\DESCRIPTION
Returns the real-space coordinates corresponding to the
\code{ix}, \code{iy}, and \code{iz} components of \var{subregion}.

%=========================== NOTES
\NOTES
This is actually a macro.

%=========================== SEE ALSO
\SEEALSO
\vref{Grid Structures}{Grid Structures}\\
\vref{Gridding}{Gridding}

\end{deftypefn}

%=============================================================================
% Reference: SubgridXYZ
%=============================================================================

\newpage
\subsection{SubgridXYZ}
\label{SubgridXYZ}

\index{\findex{SubgridXYZ}}

\begin{deftypefn}{Function}{double}{SubgridX}({Subgrid *} \var{subgrid})
\end{deftypefn}
\begin{deftypefn}{Function}{double}{SubgridY}({Subgrid *} \var{subgrid})
\end{deftypefn}
\begin{deftypefn}{Function}{double}{SubgridZ}({Subgrid *} \var{subgrid})

%=========================== DESCRIPTION
\DESCRIPTION
Returns the real-space coordinates corresponding to the
\code{ix}, \code{iy}, and \code{iz} components of \var{subgrid}.

%=========================== NOTES
\NOTES
This is actually a macro.

%=========================== SEE ALSO
\SEEALSO
\vref{Grid Structures}{Grid Structures}\\
\vref{Gridding}{Gridding}

\end{deftypefn}

%=============================================================================
% Reference: DuplicateSubregion
%=============================================================================

\newpage
\subsection{DuplicateSubregion}
\label{DuplicateSubregion}

\index{\findex{DuplicateSubregion}}

\begin{deftypefn}{Function}{Subregion *}{DuplicateSubregion}({Subregion *} \var{subregion})

%=========================== DESCRIPTION
\DESCRIPTION
Returns a duplicate of \var{subregion}.

%=========================== SEE ALSO
\SEEALSO
\vref{Gridding}{Gridding}

\end{deftypefn}

%=============================================================================
% Reference: AppendSubregion
%=============================================================================

\newpage
\subsection{AppendSubregion}
\label{AppendSubregion}

\index{\findex{AppendSubregion}}

\begin{deftypefn}{Function}{void}{AppendSubregion}({Subregion *} \var{subregion}, {SubregionArray *} \var{subregion\_array})

%=========================== DESCRIPTION
\DESCRIPTION
Appends \var{subregion} to \var{subregion\_array}.

%=========================== NOTES
\NOTES
{\bf Important:}
No new \code{Subregion} structure is created; only the pointer
is copied into \var{subregion\_array}.
If one is not careful, it is easy to free \code{Subregion} structures
accidentally.

%=========================== SEE ALSO
\SEEALSO
\vref{Gridding}{Gridding}

\end{deftypefn}

%=============================================================================
% Reference: AppendSubregionArray
%=============================================================================

\newpage
\subsection{AppendSubregionArray}
\label{AppendSubregionArray}

\index{\findex{AppendSubregionArray}}

\begin{deftypefn}{Function}{void}{AppendSubregionArray}({SubregionArray *} \var{subregion\_array\_0}, {SubregionArray *} \var{subregion\_array\_1})

%=========================== DESCRIPTION
\DESCRIPTION
Appends \var{subregion\_array\_0} to \var{subregion\_array\_1}.

%=========================== NOTES
\NOTES
{\bf Important:}
No new \code{Subregion} structures are created; only pointers
are copied into the \var{subregion\_array\_1}.
If one is not careful, it is easy to free \code{Subregion} structures
accidentally.

%=========================== SEE ALSO
\SEEALSO
\vref{Gridding}{Gridding}

\end{deftypefn}

%=============================================================================
% Reference: DuplicateSubgrid
%=============================================================================

\newpage
\subsection{DuplicateSubgrid}
\label{DuplicateSubgrid}

\index{\findex{DuplicateSubgrid}}

\begin{deftypefn}{Function}{Subgrid *}{DuplicateSubgrid}({Subgrid *} \var{subgrid})

%=========================== DESCRIPTION
\DESCRIPTION
Returns a duplicate of \var{subgrid}.

%=========================== NOTES
\NOTES
This is currently a macro (\ref{DuplicateSubregion}).

%=========================== SEE ALSO
\SEEALSO
\vref{Gridding}{Gridding}

\end{deftypefn}

%=============================================================================
% Reference: AppendSubgrid
%=============================================================================

\newpage
\subsection{AppendSubgrid}
\label{AppendSubgrid}

\index{\findex{AppendSubgrid}}

\begin{deftypefn}{Function}{void}{AppendSubgrid}({Subgrid *} \var{subgrid}, {SubgridArray *} \var{subgrid\_array})

%=========================== DESCRIPTION
\DESCRIPTION
Appends \var{subgrid} to \var{subgrid\_array}.

%=========================== NOTES
\NOTES
This is currently a macro (\ref{AppendSubregion}).

%=========================== SEE ALSO
\SEEALSO
\vref{Gridding}{Gridding}

\end{deftypefn}

%=============================================================================
% Reference: AppendSubgridArray
%=============================================================================

\newpage
\subsection{AppendSubgridArray}
\label{AppendSubgridArray}

\index{\findex{AppendSubgridArray}}

\begin{deftypefn}{Function}{void}{AppendSubgridArray}({SubgridArray *} \var{subgrid\_array\_0}, {SubgridArray *} \var{subgrid\_array\_1})

%=========================== DESCRIPTION
\DESCRIPTION
Appends \var{subgrid\_array\_0} to \var{subgrid\_array\_1}.

%=========================== NOTES
\NOTES
This is currently a macro (\ref{AppendSubregionArray}).

%=========================== SEE ALSO
\SEEALSO
\vref{Gridding}{Gridding}

\end{deftypefn}

%=============================================================================
% Reference: ConvertToSubregion
%=============================================================================

\newpage
\subsection{ConvertToSubregion}
\label{ConvertToSubregion}

\index{\findex{ConvertToSubregion}}

\begin{deftypefn}{Function}{Subregion *}{ConvertToSubregion}({Subgrid *} \var{subgrid})

%=========================== DESCRIPTION
\DESCRIPTION
Converts \var{subgrid} to a \code{Subregion *} and returns it.

%=========================== NOTES
\NOTES
This is currently a macro which simply does a cast.

%=========================== SEE ALSO
\SEEALSO
\vref{ConvertToSubgrid}{ConvertToSubgrid}\\
\vref{Gridding}{Gridding}

\end{deftypefn}

%=============================================================================
% Reference: ConvertToSubgrid
%=============================================================================

\newpage
\subsection{ConvertToSubgrid}
\label{ConvertToSubgrid}

\index{\findex{ConvertToSubgrid}}

\begin{deftypefn}{Function}{Subgrid *}{ConvertToSubgrid}({Subregion *} \var{subregion})

%=========================== DESCRIPTION
\DESCRIPTION
Converts \var{subregion} to a \code{Subgrid *} if possible,
and returns it.
\var{subregion} is not duplicated, it is modified.
If a conversion cannot be made, \code{NULL} is returned.

%=========================== SEE ALSO
\SEEALSO
\vref{ConvertToSubregion}{ConvertToSubregion}\\
\vref{Gridding}{Gridding}

\end{deftypefn}

%=============================================================================
% Reference: ProjectSubgrid
%=============================================================================

\newpage
\subsection{ProjectSubgrid}
\label{ProjectSubgrid}

\index{\findex{ProjectSubgrid}}

\begin{deftypefn}{Function}{Subregion *}{ProjectSubgrid}({Subgrid *} \var{subgrid}, int \var{sx}, int \var{sy}, int \var{sz}, int \var{ix}, int \var{iy}, int \var{iz})

%=========================== DESCRIPTION
\DESCRIPTION
Projects \var{subgrid} onto an {\em index-region}.
The index-region is the collection of index-space indices with
strides \var{sx}, \var{sy}, \var{sz}, and containing
the indices \var{ix}, \var{iy}, \var{iz}.
The base index-space is determined by the \var{subgrid}.
\var{subgrid} is not modified.

%=========================== SEE ALSO
\SEEALSO
\vref{Gridding}{Gridding}

\end{deftypefn}

%=============================================================================
% Reference: ExtractSubgrid
%=============================================================================

\newpage
\subsection{ExtractSubgrid}
\label{ExtractSubgrid}

\index{\findex{ExtractSubgrid}}

\begin{deftypefn}{Function}{Subgrid *}{ExtractSubgrid}(int \var{rx}, int \var{ry}, int \var{rz}, {Subgrid *} \var{subgrid})

%=========================== DESCRIPTION
\DESCRIPTION
Returns, if possible, the largest subgrid with resolution
\var{rx}, \var{ry}, \var{rz},
contained in \var{subgrid}

%=========================== NOTES
\NOTES
This routine is currently used in only one place and is very similar
to \code{ProjectSubgrid}, so can probably be eliminated at some point.

%=========================== SEE ALSO
\SEEALSO
\vref{ProjectSubgrid}{ProjectSubgrid}\\
\vref{Gridding}{Gridding}

\end{deftypefn}

%=============================================================================
% Reference: IntersectSubgrids
%=============================================================================

\newpage
\subsection{IntersectSubgrids}
\label{IntersectSubgrids}

\index{\findex{IntersectSubgrids}}

\begin{deftypefn}{Function}{Subgrid *}{IntersectSubgrids}({Subgrid *} \var{subgrid1}, {Subgrid *} \var{subgrid2})

%=========================== DESCRIPTION
\DESCRIPTION
Returns the intersection of \var{subgrid1} and \var{subgrid2}.
Returns \code{NULL} if none.

%=========================== NOTES
\NOTES
Assumes the index spaces for the two subgrids are legally nested
(\ref{Gridding}.

%=========================== SEE ALSO
\SEEALSO
\vref{Gridding}{Gridding}

\end{deftypefn}

%=============================================================================
% Reference: SubtractSubgrids
%=============================================================================

\newpage
\subsection{SubtractSubgrids}
\label{SubtractSubgrids}

\index{\findex{SubtractSubgrids}}

\begin{deftypefn}{Function}{SubgridArray *}{SubtractSubgrids}({Subgrid *} \var{subgrid1}, {Subgrid *} \var{subgrid2})

%=========================== DESCRIPTION
\DESCRIPTION
Subtracts the intersection of \var{subgrid1} and \var{subgrid2}
from \var{subgrid1}.

%=========================== SEE ALSO
\SEEALSO
\vref{Gridding}{Gridding}

\end{deftypefn}

%=============================================================================
% Reference: UnionSubgridArray
%=============================================================================

\newpage
\subsection{UnionSubgridArray}
\label{UnionSubgridArray}

\index{\findex{UnionSubgridArray}}

\begin{deftypefn}{Function}{SubgridArray *}{UnionSubgridArray}({SubgridArray *} \var{subgrid\_array})

%=========================== DESCRIPTION
\DESCRIPTION
Unions the subgrids in \var{subgrid\_array}.
This involves removing any duplication of grid points, resulting in
a subgrid-array of disjoint subgrids.
The routine also tries to agglomerate subgrids into bigger subgrids.

%=========================== NOTES
\NOTES
{\bf Important:} This routine ignores process numbers, and currently
sets all process values in the returned union to 0.

%=========================== SEE ALSO
\SEEALSO
\vref{Gridding}{Gridding}

\end{deftypefn}

