%=============================================================================

\chapter{Manipulating Data: \pftools{}}
\label{Manipulating Data}

\section{Introduction to the \parflow{} TCL commands (PFTCL) }
Several tools for manipulating data are provided in PFTCL command set.
Tools can be accessed directly from the TCL shell or within a \parflow{} input script.
In both cases you must first load the \parflow{} package into the TCL shell as follows:


\begin{display}\begin{verbatim}
#
# To Import the ParFlow TCL package
#
lappend auto_path $env(PARFLOW_DIR)/bin
package require parflow
namespace import Parflow::*
\end{verbatim}\end{display}

In addition to these methods xpftools provides GUI access to most of these features.  However
the simplest approach is generally to include the tools commands within a tcl script. The
following section lists all of the available ParFlow TCL commands along with detailed instructions for their use.
\S~\ref{PFTCL Commands} provides several examples of pre and post processing using the tools.  In addition, a list of
tools can be obtained by typing \code{pfhelp} into a TCL shell after importing ParFlow. Typing ¿pfhelp¿ followed by a
command name will display a detailed description of the command in question.


\section{PFTCL Commands}
\label{PFTCL Commands}
The tables that follow \ref{pftools1}, \ref{pftools2} and \ref{pftools3} provide a list of ParFlow commands with short descriptions grouped according to their function.
The last two columns in this table indicate what examples from \S~\ref{common_pftcl}, if any, the command is used in
and whether the command is compatible with a terrain following grid domain formulation.
\newpage

\small{
\begin{table} \center
\caption{List of \pftools{} commands by function.}

\begin{tabular}{ | p{3cm} | p{6cm} | p{2cm} | p{3cm} | }
\hline
	\bf{Name} & \bf{Short Description} & \bf{Examples} & \bf{Compatible with TFG?} \\ \hline
	pfhelp & Get help for PF Tools &  & X \\ \hline
	\multicolumn{4}{|c|}{Mathematical Operations }   \\ \hline
	pfcellsum & datasetx + datasety &  & X \\ \hline
	pfcelldiff & datasetx - datasety &  & X \\ \hline
	pfcellmult & datasetx   * datasety &  & X  \\ \hline
	pfcelldiv & datasetx / datasety &  & X \\ \hline
	pfcellsumconst & dataset + constant &  & X \\ \hline
	pfcelldiffconst & dataset - constant &  & X \\ \hline
	pfcellmultconst & dataset * constant &  & X \\ \hline
	pfcelldivconst & dataset / constant &  & X \\ \hline
	pfsum & Sum dataset & 7, 9 & X \\ \hline
	pfdiffelt & Element difference &  & X \\ \hline
	pfprintdiff & Print difference &  & X \\ \hline
	pfmdiff & Calculate area where the difference between two datasets is less than a threshold &  & X  \\ \hline
	pfprintmdiff & Print the locations with differences greater than a minimum threshold &  & X  \\ \hline
	pfsavediff & Save the difference between two datasets &  & X \\ \hline
	pfaxpy & y=alpha*x+y &  & X  \\ \hline
	pfgetstats & Calculate dataset statistics (min, max, mean, var, stdev) &  & X \\ \hline
	pfprintstats & Print formatted statistics &  & X \\ \hline
	pfstats & Calculate and print dataset statistics (min, max, mean, var, stdev) &  & X  \\ \hline
	\multicolumn{4}{|c|}{Calculate physical parameters}   \\ \hline
	pfbfcvel & Calculate block face centered velocity &  &   \\ \hline
	pfcvel & Calculate Darcy velocity &  &  \\ \hline
	pfvvel & Calculate Darcy velocity at cell vertices &  &  \\ \hline
	pfvmag & Calculate velocity magnitude given components &  &  \\ \hline
	pfflux & Calculate Darcy flux &  &  \\ \hline
	pfhhead & Calculate hydraulic head & 2 &  \\ \hline
	pfphead & Calculate pressure head from hydraulic head &  &  \\ \hline
	pfsattrans & calculate saturated transmissivity &  & X \\ \hline
	pfupstreamarea & Calculate upstream area &  & X \\ \hline
	pfeffectiverecharge & Calculate effective recharge &  & X \\ \hline
	pfwatertabledepth & Calculate water table from saturation &  & X \\ \hline
	pfhydrostatic & Calculate hydrostatic pressure field  &  &  \\ \hline
	pfsubsurfacestorage & Calculate total sub-surface storage & 7 & X \\ \hline
	pfgwstorage & Calculate saturated subsurface storage &  & X \\ \hline
	pfsurfacerunoff & Calculate total surface runoff & 9 & X \\ \hline
	pfsurfacestorage & Calculate total surface storage & 8 &  X\\ \hline
\end{tabular}
\label{pftools1}
\end{table}

\begin{table} \center
\caption{List of \pftools{} commands by function (cont.).}

\begin{tabular}{ | p{3cm} | p{6cm} | p{2cm} | p{3cm} | }
\hline
	\bf{Name} & \bf{Short Description} & \bf{Examples} & \bf{Compatible with TFG?} \\ \hline

\multicolumn{4}{|c|}{DEM Operations}   \\ \hline
	pfslopex & Calculate slopes in the x-direction & 5 & X  \\ \hline
	pfslopey & Calculate slope in the y-direction & 5 & X \\ \hline
	pfchildD8 & Calculate D8 child &  & X \\ \hline
	pfsegmentD8 & Calculate D8 segment lengths &  & X \\ \hline
	pfslopeD8 & Calculate D8 slopes &  & X \\ \hline
	pfslopexD4 & Calculate D4 slopes in the x-direction &  & X \\ \hline
	pfslopeyD4 & Calculate D4 slopes in the y-direction &  & X \\ \hline
	pffillflats & Fill DEM flats & 5 & X \\ \hline
	pfmovingavgdem & Fill dem sinks with moving average &  & X \\ \hline
	pfpitfilldem & Fill sinks in the dem using iterative pitfilling routine & 5 & X \\ \hline
	pfflintslawfit & Calculate Flint's Law parameters &  & X \\ \hline
	pfflintslaw & Smooth DEM using Flints Law &  & X \\ \hline
	pfflintslawbybasin & Smooth DEM using Flints Law by basin &  & X \\ \hline
	\multicolumn{4}{|c|}{Topmodel functions} \\ \hline
	pftopodeficit & Calculate TOPMODEL water deficit &  & X \\ \hline
	pftopoindex & Calculate topographic index &  & X \\ \hline
	pftopowt & Calculate watertable based on topographic index &  & X \\ \hline
	pftoporecharge & Calculate effective recharge &  & X \\ \hline
\multicolumn{4}{|c|}{Domain Operations}  \\ \hline
	pfcomputedomain & Compute domain mask & 3 & X \\ \hline
	pfcomputetop & Compute domain top & 3, 6, 8, 9 & X \\ \hline
	pfextracttop & Extract domain top & 6 & X \\ \hline
	pfcomputebottom & Compute domain bottom & 3 & X \\ \hline
	pfsetgrid & Set grid & 5 & X \\ \hline
	pfgridtype & Set grid type &  & X \\ \hline
	pfgetgrid & Return grid information &  & X \\ \hline
	pfgetelt & Extract element from domain & 10 & X \\ \hline
	pfextract2Ddomain & Build 2D domain &  & X \\ \hline
	pfenlargebox & Compute expanded dataset &  & X \\ \hline
	pfgetsubbox & Return subset of data &  & X \\ \hline
	pfprintdomain & Print domain & 3 & X \\ \hline
	pfbuilddomain & Build a subgrid array from a ParFlow database &  & X \\ \hline
	\multicolumn{4}{|c|}{Dataset operations} \\ \hline
	pflistdata & Return dataset names and labels &  & X \\ \hline
	pfgetlist & Return dataset descriptions &  & X \\ \hline
	pfprintlist & Print list of datasets and their labels &  & X \\ \hline
	pfnewlabel & Change dataset label &  & X \\ \hline
	pfnewdata & Create new dataset &  & X \\ \hline
	pfprintgrid & Print grid &  & X \\ \hline
	pfnewgrid & Set grid for new dataset &  & X \\ \hline
	pfdelete & Delete dataset &  & X \\ \hline
	pfreload & Reload dataset &  & X \\ \hline
	pfreloadall & Reload all current datasets &  & X \\ \hline
	pfprintdata & Print all elements of a dataset &  & X \\ \hline
	pfprintelt & Print a single element &  & X \\ \hline
\end{tabular}
\label{pftools2}
\end{table}


\begin{table} \center
\caption{List of \pftools{} commands by function (cont.).}

\begin{tabular}{ | p{3cm} | p{6cm} | p{2cm} | p{3cm} | }
\hline
	\bf{Name} & \bf{Short Description} & \bf{Examples} & \bf{Compatible with TFG?} \\ \hline

	\multicolumn{4}{|c|}{File Operations}  \\ \hline
	pfload & Load file & All & X \\ \hline
	pfloadsds & Load Scientific Data Set from HDF file &  & X \\ \hline
	pfdist & Distribute files  based on processor topology & 4 & X \\ \hline
	pfdistondomain & Distribute files based on domain &  & X \\ \hline
	pfundist & Undistribute files &  & X  \\ \hline
	pfsave & Save dataset & 1,2,5,6 & X \\ \hline
	pfsavesds & Save dataset in an HDF format &  & X \\ \hline
	pfvtksave & Save dataset in VTK format using DEM & X & X \\ \hline
	pfwritedb & Write the settings for a PF run to a database &  & X  \\ \hline
	\multicolumn{4}{|c|}{Solid file operations} \\ \hline
	pfpatchysolid & Build a solid file between two complex surfaces and assign user-defined patches around the edges &  & X \\ \hline
	pfsolidfmtconvert & Converts back and forth between ascii and binary formats for solid files &  & X  \\ \hline
\end{tabular}
\label{pftools3}
\end{table}
}
\clearpage

Detailed descriptions of every command are included below in alphabetical order.
Note that the required inputs are listed following each command. Commands that perform
operations on data sets will require an identifier for each data set it takes as input.
Inputs listed in square brackets are optional and do not need to be provided.

\begin{description}

\item{\begin{verbatim}pfaxpy alpha x y\end{verbatim}}
This command computes y = alpha*x+y where alpha is a scalar and x and y are identifiers
representing data sets. No data set identifier is returned upon successful completion
since data set y is overwritten.

\item{\begin{verbatim}pfbfcvel conductivity phead\end{verbatim}}
This command computes the block face centered flow velocity at every grid cell.
Conductivity and pressure head data sets are given as arguments.
The output includes x, y, and z velocity components that are appended to the Tcl result.

\item{\begin{verbatim}pfbuilddomain database\end{verbatim}}
This command builds a subgrid array given a ParFlow database that contains the domain
parameters and the processor topology.

\item{\begin{verbatim}pfcelldiff datasetx datasety mask\end{verbatim}}
This command computes cell-wise differences of two datasets (diff=datasetx-datasety).
This is the difference at each individual cell, not over the domain. Datasets must have the same dimensions.

\item{\begin{verbatim}pfcelldiffconst dataset constant mask\end{verbatim}}
This command subtracts a constant value from each (active) cell of dataset (dif=dataset - constant).

\item{\begin{verbatim}pfcelldiv datasetx datasety mask\end{verbatim}}
This command computes the cell-wise quotient of datasetx and datasety (div = datasetx/datasety).
This is the quotient at each individual cell. Datasets must have the same dimensions.

\item{\begin{verbatim}pfcelldivconst dataset constant mask\end{verbatim}}
This command divides each (active) cell of dataset by a constant (div=dataset/constant).

\item{\begin{verbatim}pfcellmult datasetx datasety mask\end{verbatim}}
This command computes the cell-wise product of datasetx and datasety (mult = datasetx * datasety).
This is the product at each individual cell. Datasets must have the same dimensions.

\item{\begin{verbatim}pfcellmultconst dataset constant mask\end{verbatim}}
This command multiplies each (active) cell of dataset by a constant (mult=dataset * constant).


\item{\begin{verbatim}pfcellsum datasetp datasetq mask\end{verbatim}}
This command computes the cellwise sum of two datasets (i.e., the sum at each
individual cell, not the sum over the domain). Datasets must have the same
dimensions.


\item{\begin{verbatim}pfcellsumconst dataset constant mask\end{verbatim}}
This command adds the value of constant to each (active) cell of dataset.

\item{\begin{verbatim}pfchildD8 dem\end{verbatim}}
This command computes the unique D8 child for all cells. Child[i,j] is the
elevation of the cell to which [i,j] drains (i.e. the elevation of [i,j]'s
child). If [i,j] is a local minima the child elevation set the elevation of [i,j].

\item{\begin{verbatim}pfcomputebottom mask\end{verbatim}}
This command computes the bottom of the domain based on the mask of active and inactive zones.
The identifier of the data set created by this operation is returned upon successful completion.

\item{\begin{verbatim}pfcomputedomain top bottom\end{verbatim}} This
 command computes a domain based on the top and bottom data sets.  The
 domain built will have a single subgrid per processor that covers the
 active data as defined by the top and bottom.  This domain will more
 closely follow the topology of the terrain than the default single
 computation domain.

A typical usage pattern for this is to start with a mask file (zeros
 in inactive cells and non-zero in active cells), create the top and
 bottom from the mask, compute the domain and then write out the domain.
 Refer to example number 3 in the following section.

\item{\begin{verbatim}pfcomputetop mask\end{verbatim}}
This command computes the top of the domain based on the mask of
 active and inactive zones.  This is the land-surface in \code{clm} or
 overland flow simulations.  The identifier of the data set created by
 this operation is returned upon successful completion.

\item{\begin{verbatim}pfcvel conductivity phead\end{verbatim}}
This command computes the Darcy velocity in cells for the conductivity data set
represented by the identifier `conductivity' and the pressure head
data set represented by the identifier `phead'.  (note: This "cell"
is not the same as the grid cells; its corners are defined by the
grid vertices.)  The identifier of the data set created by this
operation is returned upon successful completion.


\item{\begin{verbatim}pfdelete dataset\end{verbatim}}
This command deletes the data set represented by the identifier `dataset'. This
command can be useful when working with multiple datasets / time series, such as
those created when many timesteps of a file are loaded and processed.  Deleting these
datasets in between reads can help with tcl memory management.


\item{\begin{verbatim}pfdiffelt datasetp datasetq i j k digits [zero]\end{verbatim}}
This command returns the difference of two corresponding coordinates
from `datasetp' and `datasetq' if the number of digits in agreement
(significant digits) differs by more than `digits' significant
digits and the difference is greater than the absolute zero given
by `zero'.


\item{\begin{verbatim}pfdist [options] filename \end{verbatim}}
Distribute the file onto the virtual file system. This utility must be used to
create files which ParFlow can use as input. ParFlow uses a virtual file system
which allows each node of the parallel machine to read from the input file independently.
The utility does the inverse of the pfundist command. If you are using a ParFlow binary
file for input you should do a pfdist just before you do the pfrun. This command
requires that the processor topology and computational grid be set in the input
file so that it knows how to distribute the data. Note that the old syntax for
pfdist required the NZ key be set to 1 to indicate a two dimensional file but this can now be
specified manually when pfdist is called by using the optional argument -nz followed by the
number of layers in the file to be distributed, then the filename.
If the -nz argument is absent the NZ key is used by default for the processor topology.

For example,
\begin{display}
\begin{verbatim}
pfdist -nz 1 slopex.pfb
\end{verbatim}
\end{display}

\item{\begin{verbatim}pfdistondomain filename domain\end{verbatim}}
 Distribute the file onto the virtual file system based on the domain
 provided rather than the processor topology as used by pfdist.  This
 is used by the SAMRAI version of which allows for a more complicated
 computation domain specification with different sized subgrids on
 each processor and allows for more than one subgrid per processor.
  Frequently this will be used with a domain created by the
 pfcomputedomain command.


\item{\begin{verbatim}pfeffectiverecharge precip et slopex slopey dem\end{verbatim}}
This command computes the effective recharge at every grid cell based on total precipitation
minus evapotranspiration (P-ET)in the upstream area. Effective recharge is consistent with TOPMODEL
definition, NOT local P-ET.   Inputs are total annual (or average annual) precipitation (precip) at each point,
total annual (or average annual) evapotranspiration (ET) at each point, slope in the x
direction, slope in the y direction and elevation.

\item{\begin{verbatim}pfenlargebox dataset sx sy sz\end{verbatim}}
This command returns a new dataset which is enlarged to be of the
new size indicated by sx, sy and sz. Expansion is done first in the
z plane, then y plane and x plane.

\item{\begin{verbatim}pfextract2Ddomain domain\end{verbatim}} This
 command builds a 2D domain based off a 3D domain.  This can be used
 for a pfdistondomain command for Parflow 2D data (such as slopes and
 soil indices).

\item{\begin{verbatim}pfextracttop top data\end{verbatim}}
This command computes the top of the domain based on the top of the
 domain and another dataset.  The identifier of the data set created
 by this operation is returned upon successful completion.

\item{\begin{verbatim}pffillflats dem\end{verbatim}}
This command finds the flat regions in the DEM and eliminates them by
bilinearly interpolating elevations across flat region.

\item{\begin{verbatim}pfflintslaw dem c p\end{verbatim}}
This command smooths the digital elevation model dem according to Flints Law, with
Flints Law parameters specified by c and p, respectively. Flints Law relates the
slope magnitude at a given cell to its upstream contributing area: S = c*A**p. In
this routine, elevations at local minima retain the same value as in the original
dem. Elevations at all other cells are computed by applying Flints Law recursively
up each drainage path, starting at its terminus (a local minimum) until a drainage
divide is reached. Elevations are computed as:

dem[i,j] = dem[child] + c*(A[i,j]**p)*ds[i,j]

where child is the D8 child of [i,j] (i.e., the cell to which [i,j] drains according
to the D8 method); ds[i,j] is the segment length between the [i,j] and its child;
A[i,j] is the upstream contributing area of [i,j]; and c and p are constants.


\item{\begin{verbatim}pfflintslawbybasin dem c0 p0 maxiter\end{verbatim}}
This command smooths the digital elevation model (dem) using  the same approach
as "pfflints law". However here the c and p parameters are fit for each basin separately.
The Flint¿s Law parameters are calculated for the provided digital elevation model dem
using the iterative Levenberg-Marquardt method of non-linear least squares minimization,
as in "pfflintslawfit". The user must provide initial estimates of c0 and p0; results are
not sensitive to these initial values. The user must also specify the maximum number of
iterations as maxiter.

\item{\begin{verbatim}pfflintslawfit dem c0 p0 maxiter\end{verbatim}}
This command fits Flint's Law parameters c and p for the provided digital elevation
model dem using the iterative Levenberg-Marquardt method of non-linear least squares
minimization. The user must provide initial estimates of c0 and p0; results are not
sensitive to these initial values. The user must also specify the maximum number of
iterations as maxiter. Final values of c and p are printed to the screen, and a
dataset containing smoothed elevation values is returned. Smoothed elevations are
identical to running pfflintslaw for the final values of c and p. Note that dem
must be a ParFlow dataset and must have the correct grid information -- dx, dy, nx,
and ny are used in parameter estimation and Flint's Law calculations. If gridded
elevation values are read in from a text file (e.g., using pfload's simple ascii
format), grid information must be specified using the pfsetgrid command.

\item{\begin{verbatim}pfflux conductivity hhead\end{verbatim}}
This command computes the net Darcy flux at vertices for the
conductivity data set `conductivity' and the hydraulic head data
set given by `hhead'.  An identifier representing the flux computed
will be returned upon successful completion.

\item{\begin{verbatim}pfgetelt dataset i j k\end{verbatim}}
This command returns the value at element (i,j,k) in data set
`dataset'.  The i, j, and k above must range from 0 to (nx - 1), 0 to
(ny - 1), and 0 to (nz - 1) respectively.  The values nx, ny, and nz
are the number of grid points along the x, y, and z axes respectively.
The string `dataset' is an identifier representing the data set whose
element is to be retrieved.

\item{\begin{verbatim}pfgetgrid dataset\end{verbatim}}
This command returns a description of the grid which serves as the
domain of data set `dataset'.  The format of the description is given
below.
\begin{itemize}
\item{\begin{verbatim}(nx, ny, nz)\end{verbatim}
The number of coordinates in each direction.}
\item{\begin{verbatim}(x, y, z)\end{verbatim}The origin of the grid.}
\item{\begin{verbatim}(dx, dy, dz)\end{verbatim}The distance between each
coordinate in each direction.}
\end{itemize}
The above information is returned in the following Tcl list format:
{nx ny nz} {x y z} {dx dy dz}

\item{\begin{verbatim}pfgetlist dataset\end{verbatim}}
This command returns the name and description of a dataset if an argument is provided.
If no argument is given, then all of the data set names followed by their descriptions
is returned to the TCL interpreter. If an argument (dataset) is given, it should be the
it should be the name of a loaded dataset.

\item{\begin{verbatim}pfgetstats dataset\end{verbatim}}
This command calculates the following statistics for the data set represented by the
identifier ¿dataset¿:minimum, maximum, mean, sum, variance, and standard deviation.

\item{\begin{verbatim}pfgetsubbox dataset il jl kl iu ju ku\end{verbatim}}
This command computes a new dataset with the subbox starting at il, jl, kl and going to iu, ju, ku.


\item{\begin{verbatim}pfgridtype gridtype\end{verbatim}}
This command sets the grid type to either cell centered if `gridtype'
is set to `cell' or vertex centered if `gridtype' is set to `vertex'.
If no new value for `gridtype' is given, then the current value of
`gridtype' is returned.  The value of `gridtype' will be returned upon
successful completion of this command.


\item{\begin{verbatim}pfgwstorage mask porosity pressure saturation specific_storage\end{verbatim}}
This command computes the sub-surface water storage (compressible and incompressible components)
based on mask, porosity, saturation, storativity and pressure fields, similar to pfsubsurfacestorage,
but only for the saturated cells.

\item{\begin{verbatim}pfhelp [command]\end{verbatim}}
This command returns a list of pftools commands. If a command is provided it gives a detailed
description of the command and the necessary inputs.

\item{\begin{verbatim}pfhhead phead\end{verbatim}}
This command computes the hydraulic head from the pressure head
represented by the identifier `phead'.  An identifier for the
hydraulic head computed is returned upon successful completion.

\item{\begin{verbatim}pfhydrostatic wtdepth top mask\end{verbatim}}
Compute hydrostatic pressure field from water table depth

\item{\begin{verbatim}pflistdata dataset\end{verbatim}}
This command returns a list of pairs if no argument is given.  The
first item in each pair will be an identifier representing the data
set and the second item will be that data set's label.  If a data
set's identifier is given as an argument, then just that data set's
name and label will be returned.


\item{\begin{verbatim}pfload [file format] filename\end{verbatim}}
Loads a dataset into memory so it can be manipulated using the other
utilities.  A file format may precede the filename in order to
indicate the file's format.  If no file type option is given, then the
extension of the filename is used to determine the default file type.
An identifier used to represent the data set will be returned upon
successful completion.

      File type options include:
\begin{itemize}
\item{\begin{verbatim}pfb\end{verbatim}} ParFlow binary format.
Default file type for files with a `.pfb' extension.
\item{\begin{verbatim}pfsb\end{verbatim}}  ParFlow scattered binary format.
Default file type for files with a `.pfsb' extension.
\item{\begin{verbatim}sa\end{verbatim}}  ParFlow simple ASCII format.
Default file type for files with a `.sa' extension.
\item{\begin{verbatim}sb\end{verbatim}} ParFlow simple binary format.
Default file type for files with a `.sb' extension.
\item{\begin{verbatim}silo\end{verbatim}} Silo binary format.
Default file type for files with a `.silo' extension.
\item{\begin{verbatim}rsa\end{verbatim}} ParFlow real scattered ASCII format.
Default file type for files with a `.rsa' extension
\end{itemize}


\item{\begin{verbatim}pfloadsds filename dsnum\end{verbatim}}
This command is used to load Scientific Data Sets from HDF files.
The SDS number `dsnum' will be used to find the SDS you wish to load
from the HDF file `filename'.  The data set loaded into memory will
be assigned an identifier which will be used to refer to the data set
until it is deleted.  This identifier will be returned upon
successful completion of the command.


\item{\begin{verbatim}pfmdiff datasetp datasetq digits [zero]\end{verbatim}}
If `digits' is greater than or equal to zero, then this command
computes the grid point at which the number of digits in agreement
(significant digits) is fewest and differs by more than `digits'
significant digits.  If `digits' is less than zero, then the point
at which the number of digits in agreement (significant digits) is
minimum is computed.  Finally, the maximum absolute difference is
computed.  The above information is returned in a Tcl list
of the following form:
{mi mj mk sd} adiff

Given the search criteria, (mi, mj, mk) is the coordinate where the
minimum number of significant digits `sd' was found and `adiff' is
the maximum absolute difference.


\item{\begin{verbatim}pfmovingaveragedem dem wsize maxiter \end{verbatim}}
This command fills sinks in the digital elevation model dem by a standard iterative
moving-average routine. Sinks are identified as cells with zero slope in both x- and
y-directions, or as local minima in elevation (i.e., all adjacent cells have higher
elevations). At each iteration, a moving average is taken over a window of width
wsize around each remaining sink; sinks are thus filled by averaging over neighboring
cells. The procedure continues iteratively until all sinks are filled or the number
of iterations reaches maxiter. For most applications, sinks should be filled prior
to computing slopes (i.e., prior to executing pfslopex and pfslopey).


\item{\begin{verbatim}pfnewdata {nx ny nz} {x y z} {dx dy dz} label\end{verbatim}}
This command creates a new data set whose dimension is described by
the lists {nx ny nz}, {x y z}, and {dx dy dz}.  The first list,
describes the dimensions, the second indicates the origin, and the
third gives the length intervals between each coordinate along each
axis.  The `label' argument will be the label of the data set that
gets created.  This new data set that is created will have all of
its data points set to zero automatically.  An identifier for the new
data set will be returned upon successful completion.


\item{\begin{verbatim}pfnewgrid {nx ny nz} {x y z} {dx dy dz} label\end{verbatim}}
Create a new data set whose grid is described by passing three lists and a label as arguments.
The first list will be the number of coordinates in the x, y, and z directions.
The second list will describe the origin. The third contains the intervals between coordinates along each axis.
The identifier of the data set created by this operation is returned upon successful completion.

\item{\begin{verbatim}pfnewlabel dataset newlabel\end{verbatim}}
This command changes the label of the data set `dataset' to
`newlabel'.


\item{\begin{verbatim}pfphead hhead\end{verbatim}}
This command computes the pressure head from the hydraulic head
represented by the identifier `hhead'.  An identifier for the pressure
head is returned upon successful completion.

\item{\begin{verbatim}pfpatchysolid -top topdata -bot botdata -msk emaskdata [optional args] \end{verbatim}}
Creates a solid file with complex upper and lower surfaces from a top surface elevation dataset (topdata), a bottom elevation dataset (botdata), and an enhanced mask dataset (emaskdata) all of which must be passed as handles to 2-d datasets that share a common size and origin. The solid is built as the volume between the top and bottom surfaces using the mask to deactivate other regions. The ``enhanced mask" used here is a gridded dataset containing integers where all
active cells have values of one but inactive cells may be given a positive integer value that identifies
a patch along the model edge or the values may be zero. Any mask cell with value 0 is omitted from the active domain and \textit{is not} written to a patch.
If an active cell is adjacent to a non-zero mask cell, the face between the active and inactive cell is assigned to the
patch with the integer value of the adjacent inactive cell. Bottom and Top patches are always written for every active cell and the West, East, South, and North
edges are written automatically anytime active cells touch the edges of the input dataset(s). Up to 30 user defined patches can be specified using arbitrary integer values that are \textit{greater than} 1.
Note that the -msk flag may be omitted and doing so will make every cell active.

The -top and -bot flags, and -msk if it is used, MUST
each be followed by the handle for the relevant dataset. Optional argument flag-name pairs include:
\begin{itemize}
\item{-pfsol} <file name>.pfsol  (or -pfsolb <file name>.pfsolb)
\item{-vtk} <file name>.vtk
\item{-sub}
\end{itemize}

where <file name> is replaced by the desired text string. The -pfsolb option creates a compact binary solid file; pfsolb cannot currently be read directly by ParFlow but it can be converted with \textit{pfsolidfmtconvert} and full support is under development. If -pfsol (or -pfsolb) is not specified the
default name "SolidFile.pfsol" will be used. If -vtk is omitted, no vtk file will be created. The vtk attributes will contain mean patch elevations and patch IDs from the enhanced mask. Edge patch IDs are shown as negative values in the vtk.
The patchysolid tool also outputs the list of the patch names in the order they are written, which can be directly copied into a ParFlow TCL script for the list of patch names. The -sub option writes separate patches for each face (left,right,front,back), which are indicated in the output patch write order list.

Assuming \$Msk, \$Top, and \$Bot are valid dataset handles from pfload, two valid examples are:
\begin{display}
\begin{verbatim}
pfpatchysolid -msk $Msk -top $Top -bot $Bot -pfsol "MySolid.pfsol" -vtk "MySolid.vtk"
pfpatchysolid -bot $Bot -top $Top -vtk "MySolid.vtk" -sub
\end{verbatim}
\end{display}
Note that all flag-name pairs may be specified in any order for this tool as long as the required argument immediately follows the flag. To use with a terrain following grid, you will need to subtract the surface elevations from the top and bottom datasets (this makes the top flat) then add back in the total thickness of your grid, which can be done using ``pfcelldiff" and ``pfcellsumconst".


\item{\begin{verbatim}pfpitfilldem dem dpit maxiter \end{verbatim}}
This command fills sinks in the digital elevation model dem by a standard iterative
pit-filling routine. Sinks are identified as cells with zero slope in both x- and
y-directions, or as local minima in elevation (i.e., all adjacent neighbors have
higher elevations). At each iteration, the value dpit is added to all remaining
sinks. The procedure continues iteratively until all sinks are filled or the number
of iterations reaches maxiter. For most applications, sinks should be filled prior
to computing slopes (i.e., prior to executing pfslopex and pfslopey).


\item{\begin{verbatim}pfprintdata dataset\end{verbatim}}
This command executes `pfgetgrid' and `pfgetelt' in order to display
all the elements in the data set represented by the identifier
`dataset'.


\item{\begin{verbatim}pfprintdiff datasetp datasetq digits [zero]\end{verbatim}}
This command executes `pfdiffelt' and `pfmdiff' to print differences
to standard output.  The differences are printed one per line along
with the coordinates where they occur.  The last two lines displayed
will show the point at which there is a minimum number of significant
digits in the difference as well as the maximum absolute difference.


\item{\begin{verbatim}pfprintdomain domain\end{verbatim}} This command
 creates a set of TCL commands that setup a domain as specified by the
 provided domain input which can be then be written to a file for
 inclusion in a Parflow input script.  Note that this kind of domain
 is only supported by the SAMRAI version of Parflow.

\item{\begin{verbatim}pfprintelt i j k dataset\end{verbatim}}
This command prints a single element from the provided dataset given an i, j, k location.

\item{\begin{verbatim}pfprintgrid dataset\end{verbatim}}
This command executes pfgetgrid and formats its output before printing
it on the screen.  The triples (nx, ny, nz), (x, y, z), and
(dx, dy, dz) are all printed on separate lines along with labels
describing each.


\item{\begin{verbatim}pfprintlist [dataset]\end{verbatim}}
This command executes pflistdata and formats the output of that
command.  The formatted output is then printed on the screen.  The
output consists of a list of data sets and their labels one per line
if no argument was given or just one data set if an identifier was
given.


\item{\begin{verbatim}pfprintmdiff datasetp datasetq digits [zero]\end{verbatim}}
This command executes `pfmdiff' and formats that command's output
before displaying it on the screen.  Given the search criteria, a line
displaying the point at which the difference has the least number of
significant digits will be displayed.  Another line displaying the
maximum absolute difference will also be displayed.


\item{\begin{verbatim}printstats dataset\end{verbatim}}
This command executes `pfstats' and formats that command's output
before printing it on the screen.  Each of the values mentioned in the
description of `pfstats' will be displayed along with a label.


\item{\begin{verbatim}pfreload dataset\end{verbatim}}
This argument reloads a dataset. Only one arguments is required, the name of the dataset to reload.

\item{\begin{verbatim}pfreloadall\end{verbatim}}
This command reloads all of the current datasets.

\item{\begin{verbatim}pfsattrans mask perm\end{verbatim}}
Compute saturated transmissivity for all [i,j] as the sum of the
permeability[i,j,k]*dz within a column [i,j]. Currently this routine
uses dz from the input permeability so the dz in permeability must be correct.
Also, it is assumed that dz is constant, so this command is not compatible with variable dz.


\item{\begin{verbatim}pfsave dataset -filetype filename\end{verbatim}}
This command is used to save the data set given by the identifier
`dataset' to a file `filename' of type `filetype' in one of the
ParFlow formats below.

File type options include:
\begin{itemize}
\item{pfb}  ParFlow binary format.
\item{sa}  ParFlow simple ASCII format.
\item{sb}  ParFlow simple binary format.
\item{silo} Silo binary format.
\item{vis}  Vizamrai binary format.
\end{itemize}

\item{\begin{verbatim}pfsavediff datasetp datasetq digits [zero] -file filename
\end{verbatim}}
This command saves to a file the differences between the values
of the data sets represented by `datasetp' and `datasetq' to file
`filename'.  The data points whose values differ in more than
`digits' significant digits and whose differences are greater than
`zero' will be saved.  Also, given the above criteria, the
minimum number of digits in agreement (significant digits) will be
saved.

If `digits' is less than zero, then only the minimum number of
significant digits and the coordinate where the minimum was
computed will be saved.

In each of the above cases, the maximum absolute difference given
the criteria will also be saved.


\item{\begin{verbatim}pfsavesds dataset -filetype filename\end{verbatim}}
This command is used to save the data set represented by the
identifier `dataset' to the file `filename' in the format given by
`filetype'.

The possible HDF formats are:
\begin{itemize}
\item{-float32}
\item{-float64}
\item{-int8}
\item{-uint8}
\item{-int16}
\item{-uint16}
\item{-int32}
\item{-uint32}
\end{itemize}


\item{\begin{verbatim}pfsegmentD8 dem\end{verbatim}}
This command computes the distance between the cell centers of every parent cell [i,j]
and its child cell. Child cells are determined using the eight-point pour method (commonly
referred to as the D8 method) based on the digital elevation model dem. If [i,j] is a
local minima the segment length is set to zero.

\item{\begin{verbatim}pfsetgrid {nx ny nz} {x0 y0 z0} {dx dy dz} dataset\end{verbatim}}
This command replaces the grid information of dataset with the values provided.

\item{\begin{verbatim}pfslopeD8 dem\end{verbatim}}
This command computes slopes according to the eight-point pour method (commonly
referred to as the D8 method) based on the digital elevation model dem. Slopes
are computed as the maximum downward gradient between a given cell and it's lowest
neighbor (adjacent or diagonal). Local minima are set to zero; where local minima
occur on the edge of the domain, the 1st order upwind slope is used (i.e., the cell
is assumed to drain out of the domain). Note that dem must be a ParFlow dataset and
must have the correct grid information -- dx and dy both used in slope calculations.
If gridded elevation values are read in from a text file (e.g., using pfload's simple
ascii format), grid information must be specified using the pfsetgrid command. It should be noted that ParFlow uses slopex and slopey (NOT D8 slopes!) in runoff calculations.


\item{\begin{verbatim}pfslopex dem\end{verbatim}}
This command computes slopes in the x-direction using 1st order upwind
finite differences based on the digital elevation model dem. Slopes at local
maxima (in x-direction) are calculated as the maximum downward gradient to
an adjacent neighbor. Slopes at local minima (in x-direction) do not drain in
the x-direction and are therefore set to zero. Note that dem must be a
ParFlow dataset and must have the correct grid information -- dx in particular
is used in slope calculations. If gridded elevation values are read from a text
file (e.g., using pfload's simple ascii format), grid information must be
specified using the pfsetgrid command.

\item{\begin{verbatim}pfslopexD4 dem\end{verbatim}}
This command computes the slope in the x-direction for all [i,j] using a
four point (D4) method. The slope is set to the maximum downward slope to the
lowest adjacent neighbor. If [i,j] is a local minima the slope is set to zero (i.e. no drainage).


\item{\begin{verbatim}pfslopey dem\end{verbatim}}
This command computes slopes in the y-direction using 1st order upwind
finite differences based on the digital elevation model dem. Slopes at local
maxima (in y-direction) are calculated as the maximum downward gradient to
an adjacent neighbor. Slopes at local minima (in y-direction) do not drain in
the y-direction and are therefore set to zero. Note that dem must be a
ParFlow dataset and must have the correct grid information - dy in particular
is used in slope calculations. If gridded elevation values are read in from a
text file (e.g., using pfload's simple ascii format), grid information must be
specified using the pfsetgrid command.


\item{\begin{verbatim}pfslopeyD4 dem\end{verbatim}}
This command computes the slope in the y-direction for all [i,j] using a four point (D4) method.
The slope is set to the maximum downward slope to the lowest adjacent neighbor. If [i,j] is a local
minima the slope is set to zero (i.e. no drainage).


\item{\begin{verbatim}pfsolidfmtconvert filename1 filename2 \end{verbatim}}
This command converts solid files back and forth between the ascii .pfsol format and the binary .pfsolb format.
The tool automatically detects the conversion mode based on the extensions of the input file names.
The \textit{filename1} is the name of source file and \textit{filename2} is the target output file to be created or overwritten.
Support to directly use a binary solid (.pfsolb) is under development but this allows a significant reduction in file sizes.

For example, to convert from ascii to binary, then back to ascii:
\begin{display}
\begin{verbatim}
pfsolidfmtconvert "MySolid.pfsol" "MySolid.pfsolb"
pfsolidfmtconvert "MySolid.pfsolb" "NewSolid.pfsol"
\end{verbatim}
\end{display}


\item{\begin{verbatim}pfstats dataset\end{verbatim}}
This command prints various statistics for the data set represented by
the identifier `dataset'.  The minimum, maximum, mean, sum, variance,
and standard deviation are all computed.  The above values are
returned in a list of the following form:
{min max mean sum variance (standard deviation)}


\item{\begin{verbatim}pfsubsurfacestorage mask porosity pressure saturation specific_storage\end{verbatim}}
This command computes the sub-surface water storage (compressible and incompressible components) based on mask, porosity, saturation, storativity and pressure fields. The equations used to calculate this quantity are given in \S~\ref{Water Balance}. The identifier
of the data set created by this operation is returned upon successful
completion.


\item{\begin{verbatim}pfsum dataset\end{verbatim}}
This command computes the sum over the domain of the dataset.


\item{\begin{verbatim}pfsurfacerunoff top slope_x slope_y  mannings pressure\end{verbatim}}
This command computes the surface water runoff (out of the domain) based
on a computed top, pressure field, slopes and mannings roughness values.
This is integrated along all domain boundaries and is calculated at any location
that slopes at the edge of the domain point outward.  This data is in units of $[L^3 T^{-1}]$
and the equations used to calculate this quantity are given in \S~\ref{Water Balance}.
The identifier
of the data set created by this operation is returned upon successful
completion.


\item{\begin{verbatim}pfsurfacestorage top pressure\end{verbatim}}
This command computes the surface water storage (ponded water on top of the domain) based on a computed
top and pressure field. The equations used to calculate this quantity are given in \S~\ref{Water Balance}. The identifier
of the data set created by this operation is returned upon successful
completion.


\item{\begin{verbatim}pftopodeficit profile m trans dem slopex slopey recharge ssat sres porosity mask\end{verbatim}}
Compute water deficit for all [i,j] based on TOPMODEL/topographic index. For more details on methods and assumptions
refer to toposlopes.c in pftools.

\item{\begin{verbatim}pftopoindex dem sx sy\end{verbatim}}
Compute topographic index for all [i,j].  Here topographic index is defined as the total upstream area divided by the contour
length, divided by the local slope. For more details on methods and assumptions refer to toposlopes.c in pftools.


\item{\begin{verbatim}pftoporecharge riverfile nriver  trans dem sx sy\end{verbatim}}
Compute effective recharge at all [i,j] over upstream area based on topmodel assumptions and given list of river points.
Notes:  See detailed notes in toposlopes.c regarding assumptions, methods, etc. Input Notes: nriver is an integer (number
of river points) river  is an array of integers [nriver][2] (list of river indices, ordered from outlet to headwaters) is
a Databox of saturated transmissivity dem    is a Databox of elevations at each cell sx is a Databox of slopes (x-dir) --
lets you use processed slopes! sy is a Databox of slopes (y-dir) -- lets you use processed slopes!

\item{\begin{verbatim}pftopowt deficit porosity ssat sres mask top wtdepth\end{verbatim}}
Compute water depth from column water deficit for all [i,j] based on TOPMODEL/topographic index.

\item{\begin{verbatim}pfundist filename, pfundist runname\end{verbatim}}
The command undistributes a \parflow{} output file.  \parflow{} uses a
distributed file system where each node can write to its own file.
The pfundist command takes all of these individual files and collapses
them into a single file.

The arguments can be a runname or a filename.  If a runname is given
then all of the output files associated with that run are
undistributed.

Normally this is done after every pfrun command.


\item{\begin{verbatim}pfupstreamarea slope_x slope_y\end{verbatim}}
This command computes the upstream area contributing to surface runoff
at each cell based on the x and y slope values provided in datasets
\file{slope_x} and \file{slope_y}, respectively. Contributing area is computed recursively
for each cell; areas are not weighted by slope direction. Areas are returned
as the number of upstream (contributing) cells; to compute actual area, simply
multiply by the cell area (dx*dy).


\item{\begin{verbatim}pfvmag datasetx datasety datasetz\end{verbatim}}
This command computes the velocity magnitude when given three velocity
components.  The three parameters are identifiers which represent
the x, y, and z components respectively.  The identifier of the data
set created by this operation is returned upon successful completion.


\item{\begin{verbatim}pfvtksave dataset filetype filename [options]\end{verbatim}}
This command loads PFB or SILO output, reads a DEM from a file and generates
a 3D VTK output field from that \parflow{} output.

The options:
Any combination of these can be used and they can be specified in any order as long as the required elements immediately follow each option.

-var specifies what the variable written to the dataset will be called. This is followed by a text string, like "Pressure" or "Saturation" to define the name of the data that will be written to the VTK. If this isn't specified, you'll get a property written to the file creatively called "Variable". This option is ignored if you are using -clmvtk since all its variables are predefined.

-dem specifies that a DEM is to be used. The argument following -dem MUST be the handle of the dataset containing the elevations. If it cannot be found, the tool ignores it and reverts to non-dem mode. If the nx and ny dimensions of the grids don’t match, the tool will error out. This option shifts the layers so that the top of the domain coincides with the land surface defined by the DEM. Regardless of the actual number of layers in the DEM file, the tool only uses the elevations in the top layer of this dataset, meaning a 1-layer PFB can be used.

-flt tells the tool to write the data as type float instead of double. Since the VTKs are really only used for visualization, this reduces the file size and speeds up plotting.

-tfg causes the tool to override the specified dz in the dataset PFB and uses a user specified list of layer thicknesses instead. This is designed for terrain following grids and can only be used in conjunction with a DEM. The argument following the flag is a text string containing the number of layers and the dz list of actual layer thicknesses (not dz multipliers) for each layer from the bottom up such as: -tfg "5 200.0 1.0 0.7 0.2 0.1"
Note that the quotation marks around the list are necessary.

Example:
\begin{display}
\begin{verbatim}
file copy -force CLM_dem.cpfb CLM_dem.pfb

set CLMdat [pfload -pfb clm.out.clm_output.00005.C.pfb]
set Pdat [pfload -pfb clm.out.press.00005.pfb]
set Perm [pfload -pfb clm.out.perm_x.pfb]
set DEMdat [pfload -pfb CLM_dem.pfb]

set dzlist "10 6.0 5.0 0.5 0.5 0.5 0.5 0.5 0.5 0.5 0.5"

pfvtksave $Pdat -vtk "CLM.out.Press.00005a.vtk" -var "Press"
pfvtksave $Pdat -vtk "CLM.out.Press.00005b.vtk" -var "Press" -flt
pfvtksave $Pdat -vtk "CLM.out.Press.00005c.vtk" -var "Press" -dem $DEMdat
pfvtksave $Pdat -vtk "CLM.out.Press.00005d.vtk" -var "Press" -dem $DEMdat -flt
pfvtksave $Pdat -vtk "CLM.out.Press.00005e.vtk" -var "Press" -dem $DEMdat -flt -tfg $dzlist
pfvtksave $Perm -vtk "CLM.out.Perm.00005.vtk" -var "Perm" -flt -dem $DEMdat -tfg $dzlist

pfvtksave $CLMdat -clmvtk "CLM.out.CLM.00005.vtk" -flt
pfvtksave $CLMdat -clmvtk "CLM.out.CLM.00005.vtk" -flt -dem $DEMdat

pfvtksave $DEMdat -vtk "CLM.out.Elev.00000.vtk" -flt -var "Elevation" -dem $DEMdat
\end{verbatim}\end{display}

\item{\begin{verbatim}pfvvel conductivity phead\end{verbatim}}
This command computes the Darcy velocity in cells for the conductivity
data set represented by the identifier `conductivity' and the pressure
head data set represented by the identifier `phead'.  The identifier
of the data set created by this operation is returned upon successful
completion.

\item{\begin{verbatim}pfwatertabledepth top saturation \end{verbatim}}
 This command computes the water table depth (distance from top to
 first cell with saturation = 1).  The identifier of the data set
 created by this operation is returned upon successful completion.


\item{\begin{verbatim}pfwritedb runname\end{verbatim}}
This command writes the settings of parflow run to a pfidb database that
can be used to run the model at a later time. In general this command is used in lieu of the pfrun command.

\end{description}

\section{Common examples using \parflow{} TCL commands (PFTCL) }
\label{common_pftcl}
This section contains some brief examples of how to use the pftools commands (along with standard \emph{TCL} commands) to postprocess data.

\begin{enumerate}

\item Load a file as one format and write as another format.
\begin{display}\begin{verbatim}
set press [pfload harvey_flow.out.press.pfb]
pfsave $press -sa harvey_flow.out.sa

#####################################################################
# Also note that PFTCL automatically assigns
#identifiers to each data set it stores. In this
# example we load the pressure file and assign
#it the identifier press. However if you
#read in a file called foo.pfb into a TCL shell
#with assigning your own identifier, you get
#the following:

#parflow> pfload foo.pfb
#dataset0

# In this example, the first line is typed in by the
#user and the second line is printed out
#by PFTCL. It indicates that the data read
#from file foo.pfb is associated with the
#identifier dataset0.

\end{verbatim}\end{display}


\item Load pressure-head output from a file, convert to head-potential and write out as a new file.

\begin{display}\begin{verbatim}
set press [pfload harvey_flow.out.press.pfb]
set head [pfhhead $press]
pfsave $head -pfb harvey_flow.head.pfb

\end{verbatim}\end{display}



\item Build a SAMARI compatible domain decomposition based off of a mask file
\begin{display}\begin{verbatim}
#---------------------------------------------------------
# This example script takes 3 command line arguments
# for P,Q,R and then builds a SAMRAI compatible
# domain decomposition based off of a mask file.
#---------------------------------------------------------

# Processor Topology
set P [lindex $argv 0]
set Q [lindex $argv 1]
set R [lindex $argv 2]
pfset Process.Topology.P $P
pfset Process.Topology.Q $Q
pfset Process.Topology.R $R

# Computational Grid
pfset ComputationalGrid.Lower.X -10.0
pfset ComputationalGrid.Lower.Y 10.0
pfset ComputationalGrid.Lower.Z 1.0

pfset ComputationalGrid.DX 8.8888888888888893
pfset ComputationalGrid.DY 10.666666666666666
pfset ComputationalGrid.DZ 1.0

pfset ComputationalGrid.NX 10
pfset ComputationalGrid.NY 10
pfset ComputationalGrid.NZ 8

# Calculate top and bottom and build domain
set mask [pfload samrai.out.mask.pfb]
set top [pfcomputetop $mask]
set bottom [pfcomputebottom $mask]

set domain [pfcomputedomain $top $bottom]
set out [pfprintdomain $domain]
set grid\_file [open samrai_grid.tcl w]

puts $grid_file $out
close $grid_file

#---------------------------------------------------------
# The resulting TCL file samrai_grid.tcl may be read into
# a Parflow input file using ¿¿source samrai_grid.tcl¿¿.
#---------------------------------------------------------

\end{verbatim}\end{display}
\item Distributing input files before running
\label{dist example}
\begin{display}\begin{verbatim}
#--------------------------------------------------------
# A common problem for new ParFlow users is to
# distribute slope files using
# the 3-D computational grid that is
# set at the begging of a run script.
# This results in errors because slope
# files are 2-D.
# To avoid this problem the computational
# grid should be reset before and after
# distributing slope files. As follows:
#---------------------------------------------------------

#First set NZ to 1 and distribute the 2D slope files
pfset ComputationalGrid.NX                40
pfset ComputationalGrid.NY                40
pfset ComputationalGrid.NZ                1
pfdist slopex.pfb
pfdist slopey.pfb

#Reset NZ to the correct value and distribute any 3D inputs
pfset ComputationalGrid.NX                40
pfset ComputationalGrid.NY                40
pfset ComputationalGrid.NZ                50
pfdist IndicatorFile.pfb

\end{verbatim}\end{display}

\item Calculate slopes from an elevation file
\begin{display}\begin{verbatim}
#Read in DEM
set dem [pfload -sa dem.txt]
pfsetgrid {209 268 1} {0.0 0.0 0.0} {100 100 1.0} $dem

# Fill flat areas (if any)
set flatfill [pffillflats $dem]

# Fill pits (if any)
set  pitfill [pfpitfilldem $flatfill 0.01 10000]

# Calculate Slopes
set  slope_x [pfslopex $pitfill]
set  slope_y [pfslopey $pitfill]

# Write to output...
pfsave $flatfill -silo klam.flatfill.silo
pfsave $pitfill  -silo klam.pitfill.silo
pfsave $slope_x  -pfb  klam.slope_x.pfb
pfsave $slope_y  -pfb  klam.slope_y.pfb
\end{verbatim}\end{display}

\item Calculate and output the \emph{subsurface storage} in the domain at a point in time.
\begin{display}\begin{verbatim}
set saturation [pfload runname.out.satur.00001.silo]
set pressure [pfload runname.out.press.00001.silo]
set specific_storage [pfload runname.out.specific_storage.silo]
set porosity [pfload runname.out.porosity.silo]
set mask [pfload runname.out.mask.silo]

set subsurface_storage [pfsubsurfacestorage $mask $porosity \
$pressure $saturation $specific_storage]
set total_subsurface_storage [pfsum $subsurface_storage]
puts [format "Subsurface storage\t\t\t\t : %.16e" $total_subsurface_storage]
\end{verbatim}\end{display}

\item Calculate and output the \emph{surface storage} in the domain at a point in time.
\begin{display}\begin{verbatim}
set pressure [pfload runname.out.press.00001.silo]
set mask [pfload runname.out.mask.silo]
set top [pfcomputetop $mask]
set surface_storage [pfsurfacestorage $top $pressure]
set total_surface_storage [pfsum $surface_storage]
puts [format "Surface storage\t\t\t\t : %.16e" $total_surface_storage]
\end{verbatim}\end{display}

\item Calculate and output the runoff out of the \emph{entire domain} over a timestep.
\begin{display}\begin{verbatim}
set pressure [pfload runname.out.press.00001.silo]
set slope_x [pfload runname.out.slope_x.silo]
set slope_y [pfload runname.out.slope_y.silo]
set mannings [pfload runname.out.mannings.silo]
set mask [pfload runname.out.mask.silo]
set top [pfcomputetop $mask]

set surface_runoff [pfsurfacerunoff $top $slope_x $slope_y $mannings $pressure]
set total_surface_runoff [expr [pfsum $surface_runoff] * [pfget TimeStep.Value]]
puts [format "Surface runoff from pftools\t\t\t : %.16e" $total_surface_runoff]
\end{verbatim}\end{display}

\item Calculate overland flow at a point using \emph{Manning's} equation
\begin{display}\begin{verbatim}
#Set the location
set Xloc 2
set Yloc 2
set Zloc 50  #This should be a z location on the surface of your domain

#Set the grid dimension and Mannings roughness coefficient
set dx  1000.0
set n   0.000005

#Get the slope at the point
set slopex   [pfload runname.out.slope_x.pfb]
set slopey   [pfload runname.out.slope_y.pfb]
set sx1 [pfgetelt $slopex $Xloc $Yloc 0]
set sy1 [pfgetelt $slopey $Xloc $Yloc 0]
set S [expr ($sx**2+$sy**2)**0.5]

#Get the pressure at the point
set press [pfload runname.out.press.00001.pfb]
set P [pfgetelt $press $Xloc $Yloc $Zloc]

#If the pressure is less than zero set to zero
if {$P < 0} { set P 0 }
set QT [expr ($dx/$n)*($S**0.5)*($P**(5./3.))]
puts $QT
\end{verbatim}\end{display}

\end{enumerate}

%=============================================================================
%=============================================================================
