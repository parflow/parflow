%=============================================================================
% Chapter: Documentation
%=============================================================================

\chapter{Documentation}
\label{Documentation}

\parflow{} documentation is installed in the directory
\begin{display}\begin{verbatim}
$PARFLOW_HELP
\end{verbatim}\end{display}
There is a subdirectory for each manual.
In \S~\ref{Writing Manuals}, we describe how to write and install
\parflow{} manuals.
In \S~\ref{Writing HTML Files} we describe how to write and install
all other HTML files.

The utility \file{pfhelp} provides online access to some or all
of the above documents.
It  uses the environment variable \code{PARFLOW_HTML_VIEWER} (which
defaults to {\em netscape}) to determine the WWW viewer that
will be used to view the online documentation.  
The environment variable
\code{PARFLOW_HELP} is used to determine the location of the
help files.

Developers on the NMG cluster should set \code{\$PARFLOW_HELP}
as described in \S~\ref{The Developer's Environment}.
Additional developer documentation is also provided via the CCSE Web server at
\begin{display}\begin{verbatim}
/home/parflow/html/index.html
\end{verbatim}\end{display}
This site is called the {\em ParFlow Info Exchange} (PFIX).


In \S~\ref{Writing AVS Module Help}, we describe how to write and
install \parflow{} AVS module help.
This online help is made available within AVS in the same manner as
that of the standard AVS module documentation.
To access it, the following must be set:
\begin{display}\begin{verbatim}
AVS_HELP_PATH=$PARFLOW_DIR/avs/help
\end{verbatim}\end{display}

%=============================================================================
% Section: Writing Manuals
%=============================================================================

\section{Writing Manuals}
\label{Writing Manuals}

The \parflow{} manuals are written in LaTeX and converted to PDF for
easier viewing.  The source for these manuals is located in
subdirectories of the directory \file{\$PARFLOW_SRC/docs}.  Various
style files used to build the \parflow{} manuals are located in the
directory \file{\$PARFLOW_SRC/docs/}.  The files \file{html.sty}
define various commands that are part of the LaTeX2HTML package.  With
these commands, one can do things such as add hyperlinks to the online
document which are ignored by LaTeX when creating the printed
document.  The file \file{parflow.sty} define various commands and
environments to be used by \parflow{} developers when writing
documentation.  They are discussed more fully in \S~\ref{ParFlow Style
  Files}.  The CMakeText.txt file contains the CMake rules used to
build the documentation when LaTex is specified during the CMake
configure.

To write and install \parflow{} manuals, do the following:
\begin{enumerate}

\item
Modify or add new \file{.tex} files to the document.
Note: make sure that new files are added to the
\file{CMakeLists.txt}.
Note that the CMake LaTex build rules copy the files from the source docs directory to 
the build directory.   Make sure to edit the files in the source directory.

\item
Build the online and printed versions, by running
\begin{display}\begin{verbatim}
make
\end{verbatim}\end{display}
in the directory where CMake was configured.

\item
Check the printed version by looking at the generated PDF file.
\begin{display}\begin{verbatim}
<pdf-viewer> user_manual.pdf
\end{verbatim}\end{display}
Use any PDF viewer installed, {\em evince} and {\em acroread} are two common applications on Linux.

\item
Finally, use \code{cvs commit} to check the document source back
into the repository.

\end{enumerate}

%=============================================================================
% SubSection: ParFlow Style Files
%=============================================================================

\subsection{ParFlow Style Files}
\label{ParFlow Style Files}

In this section, we discuss the various commands and environments
defined in \file{parflow.sty} and \file{parflow.perl}.
Many of these were taken from the Texinfo package.

The following commands are verbatim-like commands used to format
different kinds of text:
\begin{display}\begin{verbatim}
\code{<sample-program-code>}
\file{<file-name>}
\kbd{<keyboard-input>}
\key{<keyboard-key-name>}
\samp{<literal-text>}
\end{verbatim}\end{display}
These commands handle the following special characters verbatim:
\begin{display}\begin{verbatim}
# & ~ _ ^
\end{verbatim}\end{display}
These special characters are {\em not} handled verbatim:
\begin{display}\begin{verbatim}
$ % \ { }
\end{verbatim}\end{display}
The special characters \verb@$ %@ may be printed by escaping with a \verb#\#.
The special characters \verb@# & _@ are already handled verbatim, but may
also be printed by escaping with a \verb#\#.
The reason these environments are not fully verbatim is mainly
due to limitations in the LaTeX2HTML package.

The next commands are also used to format different kinds of text,
but do not print any special characters verbatim:
\begin{display}\begin{verbatim}
\dfn{<term-being-defined>}
\var{<metasyntactic-variable>}
\end{verbatim}\end{display}

The following environments are used to format descriptions of macros,
data types, typed functions, and typed variables, respectively:
\begin{display}\begin{verbatim}
\defmac{<macro-name>}(<arguments>)
\deftp{<category>}{<name-of-type>}{<attributes>}
\deftypefn{<classification>}{<data-type>}{<name>}(<arguments>)
\deftypevr{<classification>}{<data-type>}{<name>}
\end{verbatim}\end{display}
(More details may be found in the Texinfo documentation.)
Within these \code{def*} environments, several description
subsections are available.
They are:
\begin{display}\begin{verbatim}
\DESCRIPTION
\EXAMPLE
\SEEALSO
\NOTES
\end{verbatim}\end{display}
For example, the \code{NewRegion} reference (\S~\ref{NewRegion})
in this manual was created by typing the following:
\begin{display}\begin{verbatim}
\begin{deftypefn}{Function}{Region *}{NewRegion}(int \var{size})

%=========================== DESCRIPTION
\DESCRIPTION
Creates and returns a pointer to a new \code{Region} structure
containing \var{size} pointers to new \code{SubregionArray} structures.

%=========================== SEE ALSO
\SEEALSO
\vref{Grid Structures}{Grid Structures}\\
\vref{NewSubregionArray}{NewSubregionArray}\\
\vref{AppendSubregion}{AppendSubregion}\\
\vref{AppendSubregionArray}{AppendSubregionArray}\\
\vref{Gridding}{Gridding}

\end{deftypefn}
\end{verbatim}\end{display}
The command
\begin{display}\begin{verbatim}
\vref{<label>}{<title>}
\end{verbatim}\end{display}
(for ``verbose reference'') in the example above, prints the reference
number of \code{<label>}, the \code{<title>} text in brackets, and
prints the page number in the printed manual.

Several commands and environments also exist for formatting the title
and copyright pages of the manuals.
The title page is created via the \code{TitlePage} environment.
Within this environment, the following commands are allowed:
\begin{display}\begin{verbatim}
\Title{<title}
\SubTitle{<subtitle>}
\Author{<authors>}
\end{verbatim}\end{display}
The copyright page is created via the \code{CopyrightPage} environment.

The \code{index} commands of Texinfo will take some work to mimic in
the \parflow{} style files.
For the time being, the following placeholders exist:
\begin{display}\begin{verbatim}
\cindex{<index string>}
\findex{<index string>}
\kindex{<index string>}
\pindex{<index string>}
\tindex{<index string>}
\vindex{<index string>}
\end{verbatim}\end{display}
These commands currently only change the font and are used {\em within}
the \code{index} command as in the following example of a
{\em function index} call:
\begin{display}\begin{verbatim}
\index{\findex{<index string>}}
\end{verbatim}\end{display}

There are also macros available for writing \parflow{}, \xparflow{},
and \pftools{} in a consistent manner in both the printed and online
versions of the manuals.
These commands should be invoked as follows:
\begin{display}\begin{verbatim}
\parflow{}
\xparflow{}
\pftools{}
\end{verbatim}\end{display}
Leaving the \verb#{}# characters off may cause formatting problems,
so it is strongly recommended that you invoke these commands in this
manner.

For information on how to use the command
\begin{display}\begin{verbatim}
\pfaddanchor{<unique tag>}
\end{verbatim}\end{display}
to provide help info for \parflow{} GUIs,
see \S~\ref{Providing Online Help for GUIs}.

%=============================================================================
% SubSection: Providing Online Help for GUIs
%=============================================================================

\subsection{Providing Online Help for GUIs}
\label{Providing Online Help for GUIs}

A mechanism has been developed to provide online help for \parflow{} GUIs.
The command
\begin{display}\begin{verbatim}
\pfaddanchor{<unique tag>}
\end{verbatim}\end{display}
must first be added to manual sections that are to be used as online help
pages (see file \file{files.tex} of the User's Manual for examples).
This command produces an anchor in the online html document and
nothing in the printed version.
The anchor name is created by prepending the string \code{PFAnchor}
to the tag argument of the \code{pfaddanchor} command.

After the online manual is created, the command \file{do_online}
should be executed by the manual \file{Makefile}.
This routine is located in the \file{\$PARFLOW_SRC/docs/lib}
directory.
It parses the online manuals for the string \code{name=PFAnchor},
then creates the file \file{pfanchors} which contains a table of
anchor names and the files in which they appear (note, the file
is sorted by anchor name).
In this way, \parflow{} GUIs need only search the \file{pfanchors}
files to determine the location of appropriate help information.

%=============================================================================
% Section: Writing HTML Files
%=============================================================================

\section{Writing HTML Files}
\label{Writing HTML Files}

HTML files that are created ``by hand'' (i.e. not generated by a package
such as LaTeX2HTML), are kept under resource control in the directory
\file{\$PARFLOW_SRC/docs/html}.
This directory mimics the \code{\$PARFLOW_HELP} directory.
To modify and install an HTML file, do the following:
\begin{enumerate}

\item
Check out and lock the file to be modified.

\item
Make modifications.

\item
In directory \file{\$PARFLOW_SRC/docs/html}, type
\begin{display}\begin{verbatim}
build install
\end{verbatim}\end{display}
This will copy the HTML files in the current directory tree
to the appropriate place in the \file{\$PARFLOW_HELP} directory.

\item
Finally, use \kbd{xpfci} to check in the modified source.

\end{enumerate}
Note: if you are creating new HTML files or deleting old HTML
files, you will need to modify the above \file{build} script.


%=============================================================================
% Section: Writing PFIX Files
%=============================================================================

\section{Writing PFIX Files}
\label{Writing PFIX Files}

The PFIX web pages are kept under RCS control in the
\file{~parflow/html} directory.  The \file{public} directory is for
information that is for local users (helios only servers local
machines). \file{private} is used for internal use only documents for
developers.  Items which are to be available to everyone should go on
the \parflow{} home page.

If you need to modify one of these files use the
following procedure:

\begin{enumerate}
\item Go to the directory \file{~parflow/html}.
\item Checkout (and lock) the HTML file.  If the file is already
writable it is probably being updated by someone else!
\item Make the changes you want
\item Test the new page using the PFIX link
\item Checkin the HTML file.
\end{enumerate}

%=============================================================================
% Section: Writing AVS Module Help
%=============================================================================

\section{Writing AVS Module Help}
\label{Writing AVS Module Help}

This section describes how to write AVS module help.
First, it is important that the names of module files be exactly the
same as the module name, where in the file name we replace spaces by
a \code{_} character.
This convention is used by AVS to find online help files.
AVS suggests that these names be no longer than 14 characters
for portability reasons.
We have already violated this in many cases, but should probably
attempt to do this from now on.

Assume we want to write online help for a module named ``foo~bar''
(the source for which should be in a file called \file{foo_bar.c}).
We then do the following:
\begin{enumerate}

\item
Edit the file \file{foo_bar.txt} and type in the help info.
We should try to propagate a standard format for these files,
so use an existing file as a template.

\item
Type
\begin{display}\begin{verbatim}
build install
\end{verbatim}\end{display}
This installs the online help in the directory
\file{\$PARFLOW_DIR/avs/help}.

\item
For help to work, you must have either the environment variable
\code{AVS_HELP_PATH} set appropriately (see \S~\ref{Documentation}),
or you must have a \code{HelpPath} line set in your \file{.avsrc} file.
The former is recommended since it may be set using the
\code{PARFLOW_DIR} variable.

\end{enumerate}
Of course, one should follow standard resource control system (CVS)
procedures when updating or adding new help.


%=============================================================================
% Section: (Not) Writing AVS/Express Module Help
%=============================================================================

\section{(Not) Writing AVS/Express Module Help}
\label{(Not) Writing AVS/Express Module Help}

Since AVS/Express uses a proprietary help file format, it is not possible to
write help files for your modules unless you buy a license from
Bristol Help.  There is a documentation module in AVS/Express (the lightbulb
in the Standard\_Objects/Parameters library), but I strongly advise against
using it as it is poorly implemented.  For example, you can only type a
single line of text into it, and Express seems to duplicate and move it
within your application between saves and loads of the application file.
