%=============================================================================
% Chapter: Introduction
%=============================================================================

\chapter{Introduction}
\label{Introduction}

\parflow{} (\emph{PARallel FLOW}) is an integrated hydrology 
model that simulates surface and subsurface flow.
\parflow{} \cite{Ashby-Falgout90, Jones-Woodward01, KM06, M13} 
is a parallel simulation platform that operates in three modes:\begin{enumerate}
\item
steady-state saturated; 
\item
variably saturated; 
\item
and integrated-watershed flow.
\end{enumerate}
\parflow{} is especially suitable for large scale problems on a range
of single and multi-processor computing platforms. \parflow{}
simulates saturated and variably saturated
subsurface flow in heterogeneous porous media in three spatial
dimensions using a mulitgrid-preconditioned conjugate gradient solver
\cite{Ashby-Falgout90} and a Newton-Krylov nonlinear solver
\cite{Jones-Woodward01}. \parflow{} has recently been extended to
coupled surface-subsurface flow to enable the simulation of hillslope
runoff and channel routing in a truly integrated fashion
\cite{KM06}. \parflow{} is also fully-coupled with the land surface
model \code{CLM} \cite{Dai03} as described in \cite{MM05,KM08a}.  The
development and application of \parflow{} has been on-going for more
than 20 years \cite{Meyerhoff14a, Meyerhoff14b, Meyerhoff11, Mikkelson13,
Rihani10, Shrestha14, Siirila12a,
Siirila12b, Siirila12c, Sulis10, Williams11, Williams13, Ferg10, Keyes13, 
Kollat11, Condon13a, Condon13b, M13, KRM10, KRM10, SNSMM10, DMC10, AM10,
MLMSWT10, M10, FM10, KMWSVVS10, SMPMPK10, FFKM09, KCSMMB09, MTK09, dBRM08, 
MK08b, KM08b, KM08a, MK08a, MCT08,MCK07,MWH07,
  KM06, MM05, TMCZPS05, MWT03, Teal02, WGM02, Jones-Woodward01, MCT00,
  TCRM99, TBP99, TFSBA98, Ashby-Falgout90} and resulted in some of the
most advanced numerical solvers and multigrid preconditioners for
massively parallel computer environments that are available
today. Many of the numerical tools developed within the \parflow{}
platform have been turned into or are from libraries that are now
distributed and maintained at LLNL ({\em Hypre} and {\em SUNDIALS},
for example).  An additional advantage of \parflow{} is the use of a
sophisticated octree-space partitioning algorithm to depict complex
structures in three-space, such as topography, different hydrologic
facies, and watershed boundaries. All these components implemented
into \parflow{} enable large scale, high resolution watershed
simulations. 

\parflow{} is primarily written in \emph{C}, uses a modular
architecture and contains a flexible communications layer to
encapsulate parallel process interaction on a range of platforms.
\code{CLM} is fully-integrated into \parflow{} as a module and has
been parallelized (including I/O) and is written in \emph{FORTRAN
  90/95}.  \parflow{} is organized into a main executable
\file{\emph{pfdir}/pfsimulator/parflow_exe} and a library
\file{\emph{pfdir}/pfsimulator/parflow\_lib} (where \file{\emph{pfdir}} is
the main directory location) and is comprised of more than 190
separate source files.  \parflow{} is structured to allow it to be
called from within another application (\emph{e.g.} WRF, the Weather Research 
and Forecasting atmospheric model) or as a
stand-alone application.  There is also a directory structure for the
message-passing layer \file{\emph{pfdir}/pfsimulator/amps} for the
associated tools \file{\emph{pfdir}/pftools} for \code{CLM}
\file{\emph{pfdir}/pfsimulator/clm} and a directory of test cases
\file{\emph{pfdir}/test}.
This manual describes how to use \parflow{}, and is intended for
hydrologists, geoscientists, environmental scientists and engineers. 
This manual is written assuming the reader has a basic understanding
of Linux / UNIX environments, how to compose and execute scripts in various 
programming languages (e.g. TCL), and is familiar with groundwater and 
surface water hydrology, parallel computing, and numerical modeling in general.
In Chapter~\ref{Getting Started}, we describe how to install \parflow{}.
Then, we lead the user through a simple \parflow{} run.  In
Chapter~\ref{The ParFlow System}, we describe the \parflow{} system in
more detail.  Chapter~\ref{ParFlow Files} describes the formats of the
various files used by \parflow{}.  This manual provides some overview of \parflow{}
some information on building the code, examples of scripts that solve certain classes of
problems and a compendium of keys that are set for code options. 

\parflow{} has been used in a number of research studies published in the literature. 
What follows are tables of \parflow{} references with information on topics, types of problem and
application.  \ref{pfref1}, \ref{pfref2}, \ref{pfref3} and \ref{pfref4} describe the  \\
\newpage

{\scriptsize
\begin{table}
\renewcommand{\arraystretch}{2.5}
\center
\caption{List of \parflow{} references with application and process details.}
\begin{tabular}{ l  p{1.5cm} p{2cm} p{1.5cm} p{1.5cm} | c | c | c | c }
\bf{Reference} & \bf{Coupled Model} & \bf{Application} & \bf{Scale} & \bf{Domain} & \bf{TB} & \bf{TFG} & \bf{VS} & \bf{Vdz} \\ 
\hline{}

\cite{Cui14} Cui et al. (2014) & SLIM-FAST & Model Development (nitrogen biogeochemistry) & Column-Hillslope & Idealized & X &   & X &     \\
\cite{Condon14b} Condon and Maxwell (2014) & CLM & Agriculture & Watershed & Little Washita, OK &  & X& X &     \\
\cite{Meyerhoff14b} Meyerhoff et al. (2014) & SLIM-FAST & Karst Environments & 2D & Transects in Santa Fe River Watershed &   & X &   &       \\
\cite{Condon14a} Condon and Maxwell (2014) & CLM & Agriculture & Watershed & Little Washita, OK &   & X & X &     \\
\cite{M14} Maxwell et al. (2014) & Model Comparison & Many  & Idealized  &   &   & X & X &     \\ 
\cite{Meyerhoff14a} Meyerhoff et al. (2014) & - & Stochastic runoff generation, conditioning & Hillslope & Idealized & X &   & X &     \\
\cite{Williams13} Williams et al. (2013) & WRF & Atmosphere, DART, Data Assimilation & Watershed & Idealized & X &   & X &     \\
\cite{Condon13b} Condon et al. (2013) & CLM & Subsurface Heterogeneity (land surface fluxes) & Watershed & Upper Klamath, OR & X & X & X &     \\
\cite{Condon13a} Condon and Maxwell (2013) & CLM & Agriculture & Sub-Watershed & Little Washita, OK &   & X & X &     \\
\cite{Atchley13b} Atchley et al. (2013) & SLIM-FAST; CrunchFlow & Risk Assessment & Aquifer & Idealized &   & X & X &     \\
\cite{Mikkelson13} Mikkelson et al. (2013) & CLM & Mountain Pine Beetle & Hillslope & Idealized &   & X & X &     \\
\cite{M13} Maxwell (2013) & Model Development & Idealized &   &   &   & X & X & X   \\
\cite{Atchley13a} Atchley et al. (2013) & SLIM-FAST; CrunchFlow & Risk Assessment & Aquifer & Idealized & X &   &   &     \\ 
\end{tabular}
\label{pfref1}
\end{table}

\begin{table} \center
\renewcommand{\arraystretch}{2.5}
\center
\caption{List of \parflow{} references with application and process details (cont.).}

\begin{tabular}{ l  p{1.5cm} p{2cm} p{1.5cm} p{1.5cm} | c | c | c | c }
\bf{Reference} & \bf{Coupled Model} & \bf{Application} & \bf{Scale} & \bf{Domain} & \bf{TB} & \bf{TFG} & \bf{VS} & \bf{Vdz} \\   
\hline{}
   
\cite{deRooij13} de Rooij et al. (2013) & SLIM-FAST & Model Development (surface particles) & Hillslope & Idealized  &   &   & X &     \\
\cite{Ferg12} Ferguson and Maxwell (2012) & CLM & Agriculture & Watershed & Little Washita, OK  &   &   & X &     \\
\cite{Siirila12a} Siirila and Maxwell (2012) & SLIM-FAST & Risk Assessment & Aquifer & Idealized  & X &   &   &     \\
\cite{Siirila12b} Siirila and Maxwell (2012) & SLIM-FAST & Risk Assessment & Aquifer & Idealized  & X &   &   &     \\
\cite{SNSMM10} Siirila et al. (2010) & SLIM-FAST & Risk Assessment & Aquifer & Idealized  & X &   &   &     \\
\cite{Meyerhoff11} Meyerhoff and Maxwell (2011) & - & Subsurface Heterogeneity (runoff generation) & Hillslope  & Idealized  & X &   & X &     \\
\cite{Williams11} Williams and Maxwell (2011) & WRF & Atmosphere & Watershed & Idealized & X &   & X &     \\
\cite{Ferg11} Ferguson and Maxwell (2011) & CLM & Agriculture & Watershed & Little Washita, OK  &   &   & X &     \\
\cite{DMC10} Daniels et al. (2011) & - & Regional & Streamflow & Owens Valley, CA floodplain  &   &   & X &     \\
\cite{AM10} Atchley and Maxwell (2011) & CLM & Subsurface Heterogeneity (land surface processes) & Hillslope & Golden, CO  & X &   & X &     \\
\cite{MLMSWT10} Maxwell et al. (2011) & WRF & Atmosphere & Watershed  & Little Washita, OK  &   &   & X &     \\
\cite{RMC10} Rihani et al. (2010) & CLM & Subsurface Heterogeneity (land energy fluxes) & Hillslope & Idealized  &   &   & X &     \\
\end{tabular}
\label{pfref2}
\end{table}

\begin{table} \center
\renewcommand{\arraystretch}{2.5}
\center

\caption{List of \parflow{} references with application and process details (cont.).}

\begin{tabular}{ l  p{1.5cm} p{2cm} p{1.5cm} p{1.5cm} | c | c | c | c }
\bf{Reference} & \bf{Coupled Model} & \bf{Application} & \bf{Scale} & \bf{Domain} & \bf{TB} & \bf{TFG} & \bf{VS} & \bf{Vdz} \\ 
\hline{}
   
\cite{M10} Maxwell (2010) & CLM & Subsurface Heterogeneity (infiltration) & Hillslope & Rainer Mesa (Nevada Test Site) & X &   & X &     \\
\cite{FM10} Ferguson and Maxwell (2010) & CLM & Agriculture & Watershed & Little Washita, OK &   &   & X &     \\
\cite{KMWSVVS10} Kollet et al. (2010) & CLM & Computational Scaling & 45m x 45m & Idealized & X &   & X &     \\
\cite{SMPMPK10} Sulis et al. (2010) & - & Model Comparison (CATHY) & 400m × 320m × 5m & Idealized &   &   & X &     \\
\cite{KCSMMB09} Kollet et al. (2009) & CLM & Heat Transport (ParFlowE) & Column & Wagineng, NL &   &   & X &     \\
\cite{FFKM09} Frei et al. (2009) & - & Groundwater-Surface water exchange & 2 km by 5.5 km & Consumnes River & X &   & X &     \\
\cite{MK08a} Maxwell and Kollet (2008) & CLM & Climate Change (land-energy feedbacks to groudnwater) & Watershed & Little Washita, OK &   &   & X &     \\
\cite{KM08b} Kollet and Maxwell (2008) & SLIM-FAST & Residence Time Distributions  & Watershed & Little Washita, OK &   &   & X &     \\
\cite{MK08b} Maxwell and Kollet (2008) & - & Subsurface Heterogeneity (runoff) & Hillslope & Idealized & X &   & X &     \\
\cite{KM08a} Kollet and Maxwell (2008) & CLM & Subsurface Heterogeneity (land energy fluxes) & Watershed & Little Washita, OK &   &   & X &     \\
\end{tabular}
\label{pfref3}
\end{table}

\begin{table} \center
\renewcommand{\arraystretch}{2.5}
\center

\caption{List of \parflow{} references with application and process details (cont.).}

\begin{tabular}{ l  p{1.5cm} p{2cm} p{1.5cm} p{1.5cm} | c | c | c | c }
\bf{Reference} & \bf{Coupled Model} & \bf{Application} & \bf{Scale} & \bf{Domain} & \bf{TB} & \bf{TFG} & \bf{VS} & \bf{Vdz} \\ 
\hline{}

\cite{MWH07} Maxwell et al. (2007) & particles & Subsurface Transport & Aquifer & Cape Cod, MA & X &   &   &     \\
\cite{MCK07} Maxwell et al. (2007) & ARPS, CLM & Model Development (ARPS) & Watershed & Little Washita, OK &   &   & X &     \\
\cite{KM06} Kollet and Maxwell (2006) & - & Model Development (Overland Flow) \& Subsurface Heterogeneity (shallow overland flow) & Catchment & Idealized & X &   & X &     \\
\cite{MM05} Maxwell and Miller (2005) & CLM & Model Development (CLM) & Column & Valdai, Russia &   &   & X &     \\
\cite{Ajami14} Ajami et al. (2014) & CLM & Spin Up (initial conditions) & Watershed & Ringkobing Fjord &   &   & X &     \\
\cite{Shrestha14} Shrestha et al. (2014) & COSMO-CLM & Model Development (TerrSysMP) & Watershed & Idealized; Rur catchment &   & X & X &     \\
\cite{Burger12} B¸rger et al. (2012) & ParFlow Web & Model Development (ParFlow Web)& - & - &   &   & X &     \\
\cite{K09} Kollet (2009) & CLM & Subsurface Heterogeneity (evapotranspiration) & Column & Idealized & X &   & X &     \\
\end{tabular}
\label{pfref4}
\end{table}

}
